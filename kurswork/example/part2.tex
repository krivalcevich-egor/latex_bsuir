%------------------------------------------------------------------------------
% \PART 2
%------------------------------------------------------------------------------
\chapter[Предварительное проеектирование системы]{ПРЕДВАРИТЕЛЬНОЕ ПРОЕКТИРОВАНИЕ СИСТЕМЫ}

%------------------------------------------------------------------------------
% \PART 2.1 Text
%------------------------------------------------------------------------------
\section{Разбиение системы на модули}
\hspace*{12.5 mm}Для обеспечения надёжной работы системы 
фотоаппарата на базе микроконтроллера ESP32-CAM был разбит
на несколько функциональных модулей. 
Каждый модуль отвечает за выполнение определённых задач, что 
упрощает разработку и тестирование устройства. Основные модули 
системы:

    1 Модуль инициализации камеры.
Этот модуль отвечает за конфигурацию и запуск камеры ESP32-CAM. 
В нём задаются параметры камеры, такие как частота тактирования 
(XCLK), разрешение изображения (1024x768 пикселей) и формат 
данных (JPEG). Также модуль управляет буферами кадров для 
хранения изображений, полученных с камеры.

    2 Модуль управления веб-сервером.
Данный блок реализует веб-сервер на основе ESP32s, который предоставляет 
пользователю доступ к функционалу устройства через веб-браузер. 
Он обрабатывает запросы на:
    
    - Отображение видео потока с камеры в режиме реального времени.

    - Съёмку фотографий и сохранение их на SD-карту.

    - Возврат страницы с подтверждением успешного выполнения операции.

    3 Модуль захвата и сохранения изображений.
После получения команды от веб-сервера этот модуль захватывает 
кадр с камеры, конвертирует его в JPEG и сохраняет на SD-карту. 
Каждое изображение получает уникальное имя, основанное на 
текущем времени, чтобы избежать перезаписи существующих файлов.

    4 Модуль управления Wi-Fi.
Этот модуль настраивает и управляет точкой доступа Wi-Fi, 
создаваемой микроконтроллером ESP32-CAM. Он позволяет пользователю 
подключаться к устройству через беспроводную сеть и 
взаимодействовать с веб-интерфейсом. Для этого задаются 
параметры SSID и пароля для доступа.

    5 Модуль работы с файловой системой.
Данный блок отвечает за работу с файловой системой SD-карты. 
Он инициализирует SD-карту, открывает и закрывает файлы для 
записи, а также проверяет наличие свободного места для 
сохранения изображений.

%------------------------------------------------------------------------------
% \PART 2.2 Text
%------------------------------------------------------------------------------
\section{Разработка структурной схемы устройства}
\hspace*{12.5 mm}Устройство, реализованное в данном курсовом 
проекте, состоит из трёх основных аппаратных блоков:

    1 Модуль камеры OV2640.

    2 Память на основе SD-карты. 

    3 Микроконтроллер ESP32-cam. 
    
    Модуль OV2640 обеспечивает захват изображения с 
последующим преобразованием его в цифровой формат 
для дальнейшей обработки микроконтроллером ESP32-cam.

    Для долговременного хранения 
изображений используется SD-карта на 4Гб, которая отформатирована в 
файловой системе FAT32. Этот формат широко поддерживается и 
позволяет легко работать с большими файлами. В данном проекте 
SD-карта служит для хранения снимков, захваченных камерой, в 
формате JPEG.

    Микроконтроллер ESP32-cam — это основной вычислительный 
блок системы, который управляет работой камеры, обработкой 
изображений, сохранением данных на SD-карту и организацией 
взаимодействия с пользователем через веб-интерфейс.

    Для удобства разработки и упрощения структуры программного 
обеспечения микроконтроллер ESP32-cam разделён на несколько 
модулей, каждый из которых выполняет свои специфические функции.

Схема электрическая структурная представлена в Приложении А.