%------------------------------------------------------------------------------
% \PART 5
%------------------------------------------------------------------------------
\chapter[Моделирование работы системы]
{МОДЕЛИРОВАНИЕ РАБОТЫ СИСТЕМЫ}

%------------------------------------------------------------------------------
% \PART 5.1 Text
%------------------------------------------------------------------------------
\section{Описание процесса моделирования}
\hspace*{12.5 mm}В процессе моделирования была выполнена проверка работы устройства 
на реальном оборудовании. Для этого требуемые аппаратные блоки 
были подключены к микроконтроллеру ESP32S, после чего на него была загружена разработанная прошивка.

    Для реализации процесса моделирования были использованы
следующие устройств:

    1 Микроконтроллер ESP32S;

    2 Камера OV2640;

    3 microSD-карта для хранения изображений;

    4 Программатор;

    5 Картридер для считывания данных с SD-карты.

    На рисунке~\ref{fig:device} представлены все использованные устройства.

\insertfigure{device}{pic/devices.jpg}{Используемые устройства}{9cm}

    Все аппаратные модули, такие как 
камера OV2640 и слот для SD-карты, были подключены к 
соответствующим выводам ESP32S в соответствии с принципиальной 
схемой. Также было подано питание на все компоненты для их 
корректной работы.

На рисунке~\ref{fig:yct} представлена собранная установка.

\insertfigure{yct}{pic/dev.jpg}{Собранная установка}{13cm}

    Далее было необходимо написать прошивку для микроконтроллера,
которая будет выполнять корректное взаимодействие между блоками,
а также описать веб-страницу\cite{ESP-CAM}, для удобства работы с камерой.

    Описанная прошивка микроконтроллера ESP32-cam и файлы веб-страницы
представлены в Приложении Г. А файлы веб-страниц описаны в приложении Д.

    На ESP32 была загружена прошивка, 
реализующая функции захвата и сохранения изображения, а также 
управления устройством через веб-интерфейс. Для этого устройство
через microUSB разъем программатора соединяется с компьютером.
Прошивка микроконтроллера осуществляется через ArduinoIDE.
Для этого необходимо выбрать следующие настройки среды:

\insertfigure{arduino}{pic/ArduinoIDE.png}{Настройка ArduinoIDE}{13cm}

    На рисунке~\ref{fig:esp_pc} представлен принцип подключение устройства к 
компьютеру.

\insertfigure{esp_pc}{pic/ESP-to-pc.jpg}{Подключение к компьютеру}{10.5cm}

После загрузки 
прошивки была проверена базовая работоспособность устройства, 
включая инициализацию камеры, настройку Wi-Fi соединения и 
работу с SD-картой. В результате были получены соответствующие 
сообщение в консоль.

\insertfigure{term}{pic/serial_monitor.png}{Консоль ArduinoIDEу}{10.5cm}

    Для перехода на веб страницу необходимо в браузере вписать 
адресс 192.168.4.1, что показано на рисунке~\ref{fig:google}.

\insertfigure{google}{pic/google.png}{Поиск в веб-странице}{10.5cm}

    На этапе тестирования были 
проверены основные функции устройства:

    - Захват изображения при получении команды через веб-интерфейс.

    - Сжатие изображения в формат JPEG и сохранение его на SD-карту.

    - Проверка сохранения данных с использованием картридера.

    Для тестирования необходимо было перейти на веб-страницу.
Далее перед пользователем появляется прямая-трансляция, 
благодаря которой он может отслеживать какое изображение 
будет сфотографировано. Для того чтобы осуществить съемку
необходимо нажать на кнопку <<Take Photo>>.

    На рисунке~\ref{fig:html} представлена основная страница с прямая-трансляцией
и используемой кнопкой.    

\insertfigure{html}{pic/html.png}{Основная страница с прямая-трансляцией}{12.5cm}

    После нажатия на кнопку <<Take Photo>>, осуществляется 
захват кадра, а затем оно преобразуется к требуемому
формату и разрешению и сохраняется на SD-карту. После 
успешного захвата и сохранения изображения прямая 
трансляция останавливается. Этот шаг необходим для 
освобождения ресурсов камеры и буфера, поскольку 
трансляция требует постоянного доступа к камере. 
Пользователь получает
на странице сообщение о том, что фотография успешно 
сохранена. Затем он получает возможность вернуться 
на начальную страницу через кнопку <<Go Back to Main Page>>.

На рисунке~\ref{fig:ph_done} представлена основная страница после нажатия на 
кнопку <<Take Photo>>.    

\insertfigure{ph_done}{pic/photo_done.png}{Страница после нажатия на кнопку <<Take Photo>>}{10.5cm}

    Было сделано несколько фотографий. Полученные изображения
сохраняются на SD-карте. Для того,
чтобы проверить корректность снимков, а также соответствие 
требованиям необходимо вытащить SD-карту из разъема 
микроконтроллера и подключить через картридер к компьютеру.

На рисунке~\ref{fig:res} представлены результаты фотографирования изображения.    

\insertfigurescustom{pic/res0.jpg}{pic/res1.jpg}{8cm}{Результаты фотографирования}{res}{sidebyside}

    Фотографии были сохранены корректно. Для того, чтобы 
проверить соответствие параметрам необходимо перейти в
информацию о изображении, где показано разрешение и
формат изображения.

На рисунке~\ref{fig:param} представлены параметры одного из изображений.    

\insertfigure{param}{pic/parameters.png}{Параметры изображения}{10.5cm}

    Таким образом, система отработала корректно. Мы получили 
изображение в требуемом разрешении(1024x768) и формате(JPEG).
