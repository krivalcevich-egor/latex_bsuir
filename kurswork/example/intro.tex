%------------------------------------------------------------------------------
% \Introduction
%------------------------------------------------------------------------------
\chapter*{\vspace*{-\baselineskip}\begin{center}ВВЕДЕНИЕ\end{center}\vspace*{-\baselineskip*2}}
\addcontentsline{toc}{chapter}{\hspace*{-1em}Введение}\par

%------------------------------------------------------------------------------
% Text
%------------------------------------------------------------------------------
\hspace*{12.5 mm}Курсовой проект посвящён разработке цифрового устройства на
базе микроконтроллера с использованием систем автоматизированного 
проектирования и тестирование на микроконтроллере.

    Цифровые устройства на базе микроконтроллеров находят широкое применение в 
различных областях, от бытовой электроники до промышленных систем 
автоматизации. Микроконтроллеры являются ключевыми элементами управления 
многими электронными устройствами благодаря их универсальности, компактности и 
низкой стоимости. Разработка таких устройств требует знания как 
программирования, так и основ аппаратного обеспечения.

    Целью курсового проекта является получение практических навыков 
проектирования цифровых устройств на базе микроконтроллеров.

    В рамках данного курсового проекта по дисциплине «Микропроцессорные
средства и системы» рассматривается реализация простого фотоаппарата на базе 
микроконтроллера ESP32-CAM. Этот микроконтроллер оснащен встроенной камерой и 
поддерживает программирование через Arduino IDE. Проект направлен на изучение 
процессов проектирования, программирования и тестирования цифровых устройств, 
а также на закрепление навыков работы с микроконтроллерами.

    Реализация фотоаппарата включает в себя захват изображений, их обработку и 
сохранение. Итогом работы станет устройство, способное выполнять функции 
цифровой камеры с базовыми возможностями.