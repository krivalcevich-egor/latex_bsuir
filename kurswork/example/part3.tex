%------------------------------------------------------------------------------
% \PART 3
%------------------------------------------------------------------------------
\chapter[Проектирование аппаратных средств системы]
{ПРОЕКТИРОВАНИЕ АППАРАТНЫХ СРЕДСТВ СИСТЕМЫ}

%------------------------------------------------------------------------------
% \PART 3.1 Text
%------------------------------------------------------------------------------
\section{Выбор аппаратных модулей}\par
\hspace*{12.5 mm}При выборе аппаратных модулей для реализации 
устройства было учтено несколько ключевых факторов: 
совместимость с микроконтроллером ESP32S, энергоэффективность, 
функциональные возможности и простота интеграции. 

    Камера OV2640 была выбрана за её гибкость и широкую 
популярность в проектах с микроконтроллерами. OV2640 
поддерживает различные разрешения, что позволяет легко 
настраивать качество изображения. Она имеет встроенную функцию 
аппаратного сжатия JPEG, что облегчает работу микроконтроллера, 
позволяя ему сосредоточиться на других задачах, таких как 
обработка данных и управление беспроводной связью. Камера легко 
интегрируется с ESP32S благодаря совместимости с 
интерфейсами микроконтроллера.

    Формат JPEG (Joint Photographic Experts Group) — это один из 
наиболее популярных форматов для сжатия изображений с 
потерями, широко используемый в цифровой фотографии и 
веб-дизайне. Основная особенность JPEG заключается в том, 
что он использует сжатие с потерями, что позволяет 
значительно уменьшить размер изображения за счёт отбрасывания 
малозаметной информации. JPEG использует дискретное 
косинусное преобразование (DCT), которое разделяет 
изображение на блоки, а затем кодирует каждый блок в 
частотной области, уменьшая объем данных за счет отбрасывания 
менее значимых деталей.

    Формат был разработан группой JPEG в 1992 году и с тех пор 
стал стандартом для хранения и передачи растровых изображений, 
благодаря высокому уровню сжатия при приемлемом уровне 
качества изображения\cite{GonW}.

    Для долговременного хранения данных 
используется SD-карта. В проекте применяется SD-карта, 
отформатированная в файловой системе FAT32, что обеспечивает 
стабильную работу с файлами большого размера, такими как 
изображения в формате JPEG. Выбор данного модуля обусловлен 
необходимостью хранения значительных объёмов данных  
и удобством доступа к ним.

    Основным вычислительным 
блоком системы является ESP32S, который сочетает в себе 
возможности работы с камерой, Wi-Fi подключение и работу с 
файловой системой. ESP32s был выбран из-за его 
интегрированной поддержки работы с изображениями и 
возможностью быстрой передачи данных через беспроводные сети, 
что упрощает взаимодействие устройства с пользователем.

    Система FAT берет своё начало в конце 1970-х годов, когда IBM и 
Microsoft начали сотрудничество над созданием операционной системы 
для первых персональных компьютеров. Первоначальная версия была 
представлена как FAT12. Она позволила организовывать информацию на 
дисках емкостью до 16 МБ, что являлось значительным шагом в то время.

    FAT12: Первая версия файловой системы, разработанная для 
поддержки небольших носителей данных, таких как гибкие диски. 
Основное новшество заключалось в таблице размещения файлов (File 
Allocation Table, FAT), которая обеспечивала контроль за свободными 
и занятыми блоками на диске.

    FAT16: С развитием технологий и увеличением объема жестких 
дисков появилась необходимость в расширении возможностей файловой 
системы. FAT16 позволила работать с дисками объемом до 2 ГБ. Это 
нововведение стало значительным прогрессом и обеспечило дальнейшее 
развитие компьютерных технологий.

    Файловая система FAT32 была представлена в 1996 году с релизом 
Windows 95 OSR2. Она была ответом на вызовы времени, связанные с 
потребностью хранения больших объемов данных на компактных носителях. 
Главными особенностями системы стали:

    - Увеличение максимального размера дискового раздела до 2 ТБ;

    - Совместимость с предыдущими версиями FAT, что обеспечивало плавный переход для пользователей;

    - Улучшенная управляемость файлами, благодаря экономному использованию пространства\cite{fat}.


%------------------------------------------------------------------------------
% \PART 3.2
%------------------------------------------------------------------------------
\section{Разработка принципиальной схемы устройства}\par
\hspace*{12.5 mm}В проекте была разработана принципиальная 
схема устройства, которая включает ключевые аппаратные 
компоненты, такие как микроконтроллер ESP32S, камера OV2640 и 
слот для SD-карты. Эти элементы образуют систему, 
обеспечивающую захват изображений, их обработку и хранение.
Далее подробно описаны функции и подключения каждого из 
элементов схемы.
\bf{3.2.1} \normalfont{Микроконтроллер ESP32S:} Микроконтроллер ESP32S является центральным 
узлом системы, который управляет всеми процессами и 
обеспечивает взаимодействие между периферийными устройствами. 
ESP32S обладает встроенным модулем Wi-Fi, что позволяет 
устройству поддерживать беспроводную связь для передачи данных 
пользователю через веб-интерфейс. На рисунке~\ref{fig:esp32s} представлена
распиновка ESP32S\cite{ESP-S}.

\insertfigure{esp32s}{pic/esp32s.png}{Микроконтроллер ESP32S}{9cm}

    - Питание: Напряжение питания подаётся на выводы VDD33 и 
GND, обеспечивая стабильную работу ESP32S.

    - Подключение к камере: Для взаимодействия с камерой OV2640
используются специальные пины ESP32S:

    - Линии данных камеры (Y7–Y9 и Y2–Y6) 
подключаются к соответствующим цифровым выводам 
микроконтроллера, что позволяет получать цифровое изображение.  
    - Сигнальные линии VSYNC и HREF служат для 
синхронизации кадров и строк, упрощая процесс передачи данных.

    - Тактовый сигнал (PCLK) подаётся на 
камеру для обеспечения её синхронизации с микроконтроллером.

    Подключение к SD-карте:
    ESP32S поддерживает интерфейс SDIO для взаимодействия со с
лотом для SD-карты:
    - Линии данных (DATA0–DATA3) обеспечивают двунаправленную 
передачу информации между микроконтроллером и картой памяти.

    - Линия команд (CMD) используется для передачи управляющих 
сигналов, таких как команды чтения и записи.

    - Тактовый сигнал (CLK) синхронизирует обмен данными между 
ESP32S и SD-картой, позволяя достигать высокой скорости 
передачи.

\bf{3.2.2} \normalfont Камера OV2640 отвечает за захват 
изображений, которые затем передаются на микроконтроллер для 
дальнейшей обработки и сохранения. Эта камера способна 
генерировать изображения в различных разрешениях, и она часто 
используется в IoT-проектах благодаря своей компактности и 
поддержке стандарта JPEG для сжатия изображений.
На рисунке~\ref{fig:ov} представлена
распиновка OV2640\cite{OV}.

\insertfigure{ov}{pic/ov2640.png}{Камера OV2640}{16cm}

    Камера подключается к ESP32S 
через цифровые линии данных и сигнальные линии:

    - Линии данных (Y7–Y9 и Y2–Y6) 
используются для передачи каждого пикселя изображения.

    - Линии VSYNC и HREF обеспечивают 
синхронизацию, указывая начало и окончание кадра и строки.

    - Линия тактового сигнала (PCLK) 
контролирует синхронизацию передачи данных, определяя скорость 
обмена данными между камерой и микроконтроллером.

    - Питание камеры: Для работы камеры 
необходимо подключение к выводам DOVDD и DVDD. Питание камеры 
стабилизируется с помощью фильтрующих конденсаторов, таких как 
C1 и C3, что уменьшает влияние высокочастотных шумов и 
обеспечивает стабильную работу устройства.

\bf{3.2.3} \normalfont Слот для SD-карты служит для 
долговременного хранения изображений, захваченных камерой. 
Использование SD-карты в качестве хранилища позволяет сохранять 
изображения в формате JPEG, что обеспечивает эффективное 
использование памяти и возможность последующего доступа к 
данным.
На рисунке~\ref{fig:sd} представлена
распиновка cлота для SD-карты\cite{ESP-Wrover}.

\insertfigure{sd}{pic/sd_slot.png}{Слот для SD-карты}{3cm}

    Для взаимодействия с картой памяти 
применяется интерфейс SDIO, который состоит из нескольких линий:

    - Линии данных (DATA0–DATA3) позволяют 
передавать данные между ESP32S и SD-картой.

    - Линия команд (CMD) используется для 
отправки команд на SD-карту, таких как команды на запись или 
чтение.

    - Линия тактового сигнала (CLK) 
синхронизирует передачу данных, что позволяет микроконтроллеру 
быстро и точно обмениваться информацией с картой памяти.

    - Форматирование SD-карты: SD-карта 
отформатирована в файловую систему FAT32, что обеспечивает 
совместимость с микроконтроллером и позволяет хранить файлы 
различных размеров. FAT32 также поддерживает долговременное 
хранение данных и высокую скорость доступа, что важно для 
эффективной работы устройства.

\bf{3.2.4} \normalfont Для обеспечения стабильной работы всех 
активных элементов схемы были добавлены пассивные компоненты, 
такие как конденсаторы и резисторы.

    Конденсаторы (C1, C3, C5, C6): Фильтрующие конденсаторы 
номиналом 0.1 мкФ используются для сглаживания высокочастотных 
шумов на линиях питания. Они улучшают стабильность и надёжность 
устройства, предотвращая скачки напряжения, которые могут 
повлиять на работу чувствительных компонентов, таких как 
микроконтроллер и камера.

    Резисторы (R1, R2, R3): Резисторы с номиналами 1 кОм, 4.7 кОм 
и 10 кОм обеспечивают согласование уровней сигналов и защиту 
компонентов от перегрузки. Они также используются для ограничения 
тока на линиях связи между ESP32S и периферийными устройствами, 
предотвращая возможное повреждение компонентов при высоких 
токах.

Схема электрическая принципиальная представлена в Приложении Б.
