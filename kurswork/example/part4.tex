%------------------------------------------------------------------------------
% \PART 4
%------------------------------------------------------------------------------
\chapter[Проектирование программного обеспечения]
{ПРОЕКТИРОВАНИЕ ПРОГРАММНОГО ОБЕСПЕЧЕНИЯ}

%------------------------------------------------------------------------------
% \PART 4.1 Text
%------------------------------------------------------------------------------
\section{Разработка схемы алгоритма работы системы}\par
\hspace*{12.5 mm}В данной части проекта разработан алгоритм 
работы системы в целом, который включает основные этапы 
взаимодействия компонентов устройства. Алгоритм работы системы 
описывает порядок выполнения операций по захвату изображения, 
его обработке и сохранению на SD-карту, а также организацию 
взаимодействия с пользователем через веб-интерфейс.

    Сначала система выполняет инициализацию микроконтроллера 
ESP32, настройку Wi-Fi соединения и модулей камеры и SD-карты. 
Далее происходит захват изображения с камеры, его обработка и 
сохранение на SD-карту. В завершение система переходит в режим 
ожидания команд от пользователя через веб-интерфейс, что 
позволяет загружать и просматривать сохранённые изображения.

Алгоритм работы системы описывает общий процесс, от инициализации до выполнения команд.

Выполняется настройка ESP32, проверка подключения камеры и SD-карты.

Переход в режим точки доступа, установка SSID и пароля.
Проверка успешности подключения.

Затем осуществляется подключение и настройка камеры OV2640.
Проверка наличия и правильности подключения SD-карты.

Далее система ожидает запросов от пользователя через веб-интерфейс.

Если пришла команда — <<Take photo>>, перейти к процессу захвата, обработки и сохранения изображения.

А далее осуществляется переход к режиму обновления страницы. 
При получении команды — <<Go Back to Main Page>> происходит обновление
и возврат в режим ожидания.

Схема алгоритма работы системы представлена в Приложении В.

%------------------------------------------------------------------------------
% \PART 4.2 Text
%------------------------------------------------------------------------------
\section{Разработка схемы алгоритма работы программы}
\hspace*{12.5 mm}На данном этапе была разработана схема 
алгоритма работы программы, управляющей микроконтроллером 
ESP32. Основной алгоритм программы включает следующие этапы:

1. Инициализация системы: Программа выполняет начальную 
настройку микроконтроллера, устанавливает соединение по Wi-Fi 
и инициализирует модули камеры и SD-карты.

2. Захват изображения: После инициализации программа 
отправляет команду на захват изображения с камеры OV2640.

3. Обработка изображения: Программа выполняет сжатие 
изображения в формат JPEG, что позволяет уменьшить размер 
файла для хранения.

4. Сохранение на SD-карту: Обработанное изображение 
сохраняется на SD-карту в формате JPEG.

5. Организация веб-интерфейса: Программа активирует 
веб-сервер, через который пользователь может получить доступ к 
сохранённым изображениям.

6. Ожидание команд пользователя: Программа переходит в 
режим ожидания и обрабатывает команды пользователя, такие как 
перезагрузка страницы и снимок.

Алгоритм работы программы обеспечивает надёжное выполнение 
задач по захвату, обработке и хранению изображений, а также 
удобное взаимодействие пользователя с системой через веб-интерфейс.

Алгоритм работы программы детализирует действия, выполняемые 
программой при захвате и сохранении изображений, а также при 
взаимодействии через веб-интерфейс.

Схема алгоритма работы программы представлена в Приложении В.