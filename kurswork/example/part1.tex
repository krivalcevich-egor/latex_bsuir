%------------------------------------------------------------------------------
% \PART 1
%------------------------------------------------------------------------------
\chapter[Анализ технического задания]
{АНАЛИЗ ЗАДАЧИ. ФУНКЦИОНАЛЬНАЯ СПЕЦИФИКАЦИЯ СИСТЕМЫ}

%------------------------------------------------------------------------------
% \PART 1.1 Text
%------------------------------------------------------------------------------
\section{Анализ технического задания}\par
\hspace*{12.5 mm}В данном курсовом проекте реализуется цифровое устройство в виде 
фотоаппарата на базе микроконтроллера ESP32-CAM.

Фотоаппарат построен на базе микроконтроллера ESP32-CAM, который 
представляет собой мощное и универсальное решение для разработки 
встроенных систем и устройств интернета вещей (IoT). Этот 
микроконтроллер интегрирует в себе множество функций, что делает его 
идеальным для реализации фотосъемки и хранения данных. В частности, 
ESP32-CAM оснащен модулем Wi-Fi, что позволяет осуществлять 
беспроводное управление устройством. Эта функция открывает 
возможности для удаленного доступа к камере, управления съемкой, а 
также передачи изображений на внешние устройства или в облачные 
сервисы.

Одной из ключевых особенностей этого устройства является возможность 
подключения камеры OV2640. Эта камера обеспечивает высокое качество 
изображений и поддерживает различные разрешения, что позволяет 
адаптировать устройство под разные задачи.

Еще одной важной характеристикой ESP32-CAM является поддержка работы 
с microSD-картой. Это позволяет сохранять полученные изображения на 
внешнюю память, что не только увеличивает объем доступного хранилища, 
но и упрощает процесс передачи данных. Таким образом можно 
легко извлекать карты памяти для переноса данных на другие 
устройства, что особенно полезно в условиях ограниченного доступа к 
сети.

В ходе реализации проекта предполагается разработка и интеграция 
программного и аппаратного обеспечения. Программное обеспечение 
будет включать в себя прошивку для управления камерой и Wi-Fi, а 
также интерфейсы для пользователя, обеспечивающие простоту и 
удобство в использовании. Аппаратное обеспечение, в свою очередь, 
будет включать в себя схемы подключения всех компонентов, 
необходимых для корректной работы фотоаппарата.

Проект нацелен на создание функционального устройства, 
которое не только отвечает требованиям технического задания, но и 
предоставляет пользователю широкий спектр возможностей для 
фотосъемки, хранения и передачи изображений. Все эти аспекты будут 
проработаны в рамках проекта, чтобы обеспечить надежность и 
эффективность работы фотоаппарата на базе ESP32-CAM.

%------------------------------------------------------------------------------
% \PART 1.2 Text
%------------------------------------------------------------------------------
\section{Функциональная спецификация системы}\par
\hspace*{12.5 mm}Технические требования к данному устройству определяют его 
основные функциональные возможности. Фотоаппарат должен 
выполнять следующие задачи:

1. Фотографирование по нажатию кнопки. Устройство должно уметь 
обрабатывать сигнал от кнопки, инициирующей процесс 
захвата изображения с камеры ESP32-CAM,

2. Преобразование изображения в формат JPEG с разрешением 
1024x768 пикселей. После захвата изображения оно должно быть 
преобразовано в формат JPEG для эффективного хранения и 
последующей обработки. Разрешение 1024x768 пикселей выбрано 
для обеспечения баланса между качеством изображения и размером 
файла,

3. Сохранение изображения на SD-карту. После преобразования 
изображение должно быть записано на карту памяти формата SD, 
подключённую к микроконтроллеру, что позволит пользователю 
сохранять фотографии для дальнейшего просмотра или переноса на 
другие устройства.
