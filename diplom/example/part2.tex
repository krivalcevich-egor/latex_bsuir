%------------------------------------------------------------------------------
% \PART 2
%------------------------------------------------------------------------------
\chapter[Анализ технического задания]{АНАЛИЗ ТЕХНИЧЕСКОГО ЗАДАНИЯ}

%------------------------------------------------------------------------------
% \PART 2.1 Text
%------------------------------------------------------------------------------
\section{Анализ требований к нейронной сети}
\hspace*{12.5 mm}Так как задача классификации изображений требует высокой 
точности и скорости работы, необходимо определить ключевые критерии, которым 
должна соответствоватьмодель:

  1 Нейросеть должна демонстрировать точность классификации, сравнимую с 
современными методами машинного обучения. В частности, требуется достигать 
точности не менее 97\% на тестовом наборе данных MNIST.\@ При этом важно, чтобы
сеть успешно обрабатывала различные вариации рукописных цифр, включая наклон, 
смещение и различную толщину линий.

  2 Для реализации на аппаратном уровне необходимо минимизировать количество 
параметров модели, сохраняя при этом высокую точность. Это особенно важно для 
устройств с ограниченными ресурсами, таких как встраиваемые системы.

  3 Ограниченные ресурсы памяти требуют компактного хранения параметров модели,
а также эффективного использования памяти во время вычислений. Использование 
квантования весов или сжатых представлений поможет снизить потребление памяти.

  4 Для получения вероятностного распределения классов используется функция 
активации softmax, позволяющая интерпретировать выходные значения сети в виде 
вероятностей принадлежности входного изображения к определенному классу.

%------------------------------------------------------------------------------
% \PART 2.2 Text
%------------------------------------------------------------------------------
\section{Анализ требований к аппаратной реализации}
\hspace*{12.5 mm}Аппаратная реализация нейронной сети требует учета ряда важных
факторов, которые определяют эффективность работы модели в ограниченной 
вычислительной среде.

Основные аппаратные требования включают:

  1 Для работы на FPGA необходимо эффективно использовать аппаратные ресурсы, 
такие как DSP-блоки, BRAM (блоки встроенной памяти) и логические элементы.

  2 Встроенные системы часто работают в условиях ограниченного 
энергопотребления, поэтому важно, чтобы архитектура сети была 
энергоэффективной.

  3 Архитектура должна быть масштабируемой, чтобы можно было изменять 
количество слоев или нейронов в зависимости от доступных аппаратных ресурсов. 
Кроме того, должна быть возможность модификации модели для адаптации под новые
задачи.

Основной задачей является выбор оптимальной архитектуры сети и методов 
вычислений, которые обеспечат баланс между производительностью и потреблением 
ресурсов.

%------------------------------------------------------------------------------
% \PART 2.3 Text
%------------------------------------------------------------------------------
\section{Выбор и обоснование метода решения задачи}
\hspace*{12.5 mm}Для решения задачи распознавания рукописных цифр 
рассматриваются различные подходы, каждый из которых имеет свои преимущества и 
ограничения:

Полносвязные нейронные сети  – обладают простой структурой и легко реализуются, 
однако требуют большого количества параметров. Это делает их менее подходящими 
для аппаратной реализации, так как объем необходимых вычислений и памяти растет
пропорционально числу нейронов в слоях.

Сверточные нейронные сети – обеспечивают высокую точность за счет эффективного 
извлечения признаков из изображений. Они используют свертки, позволяющие 
минимизировать количество параметров по сравнению с полносвязными сетями. 

Рекуррентные нейронные сети – могут использоваться для последовательной 
обработки данных, однако их применение к изображениям ограничено. Такие 
сети лучше подходят для обработки временных рядов и речи, но могут быть 
адаптированы к изображениям, например, в архитектурах LSTM или GRU.\@

Трансформеры – показывают высокую производительность в задачах обработки 
естественного языка и компьютерного зрения, но требуют значительных 
вычислительных ресурсов. Несмотря на их преимущества, высокая сложность и 
потребление памяти делают их менее подходящими для работы на FPGA.\@

Гибридные архитектуры – могут комбинировать различные методы, такие как CNN и 
рекуррентные слои, что позволяет достичь оптимального баланса между точностью и
вычислительными затратами.

LST2D архитектуры — основываются на архитектуре полносвязных слоев, что 
позволяет использовать минимальные вычислительные ресурсы.

На основании анализа требований к аппаратной реализации и доступных архитектур,
предлагается использовать обучаемое двумерное разделимое преобразование (LST), 
которое можно рассматривать как новый тип вычислительного слоя для построения 
архитектуры нейронной сети\cite{LST2D}. Данная структура обеспечивает высокую 
точность классификации изображений благодаря эффективному представлению 
пространственных зависимостей и позволяет сократить вычислительные затраты 
за счет оптимизированной структуры слоев.

Основное внимание уделяется разработке IP-ядра, реализующего LST с 
оптимизированными вычислениями, что позволит эффективно использовать нейросеть
в FPGA для задач классификации изображений.

Для реализации поставленных задач необходимо разработать структурную схему 
устройства. Описать алгоритм работы с учетом всех вышеперечисленных функций. 
По алгоритму описать модель на HDL языках.