%------------------------------------------------------------------------------
% \Introduction
%------------------------------------------------------------------------------
\chapter*{\vspace*{-\baselineskip}\begin{center}ВВЕДЕНИЕ\end{center}\vspace*{-\baselineskip*2}}
\addcontentsline{toc}{chapter}{\hspace*{-1em}Введение}\par

%------------------------------------------------------------------------------
% Text
%------------------------------------------------------------------------------
\hspace*{12.5 mm}Распознавание рукописных цифр является одной из ключевых 
задач в области обработки изображений и компьютерного зрения. Оно находит 
применение в различных сферах, таких как банковские системы, почтовые службы, 
идентификация документов и автоматизация процессов ввода данных. Современные 
технологии позволяют решать эту задачу с высокой точностью благодаря 
использованию методов машинного обучения и нейронных сетей.

В последние годы значительный интерес вызывает аппаратная реализация нейронных 
сетей, поскольку программные решения, работающие на центральных процессорах 
(CPU) и графических процессорах (GPU), не всегда обеспечивают требуемую 
скорость обработки и энергоэффективность. В связи с этим использование 
специализированных аппаратных ускорителей на базе FPGA (Field-Programmable 
Gate Array) или ASIC (Application-Specific Integrated Circuit) становится 
актуальной задачей. Аппаратная реализация нейронной сети в виде IP-ядра 
позволяет значительно повысить производительность и снизить задержки обработки 
данных, что особенно важно для встроенных систем.

При реализации нейронных сетей на базе CPU и GPU как правило используются 
стандартизированные типы данных (чаще всего числа с плавающей запятой одинарной
точности, реже целочисленные типы). При реализации на базе FPGA появляется 
возможность использовать для представления параметров нейронных сетей типов 
данных, обеспечивающих различную точность. Причем выбор точности представления 
напрямую будет влиять на аппаратные затраты\cite{ITS2024}. 

Целью данного дипломного проекта является разработка IP-ядра нейронной сети 
прямого распространения для распознавания рукописных цифр. В данном проекте 
используется обученное двумерное разделяемое преобразование (LST2D), которое 
рассматривается как новый тип слоя нейронной сети. Он следует концепции 
распределения веса. Проектирование IP-ядра нейронной сети прямого 
распространения производилось в несколько этапов, отраженных в структуре 
пояснительной записки данного дипломного проекта.

Первоначально был проведён анализ существующих методов и архитектур нейронных 
сетей, применяемых для задач распознавания рукописных цифр. Рассматривались 
различные подходы, включая классические многослойные персептроны, свёрточные 
и рекуррентные нейронные сети. Итоги данного этапа изложены в первом разделе 
пояснительной записки.

На следующем этапе выполнен анализ технического задания, в результате которого 
была выбрана архитектура нейронной сети, наиболее подходящая для аппаратной 
реализации. Основные критерии выбора включают компактность модели и 
вычислительную эффективность. Результаты этого анализа представлены во втором 
разделе пояснительной записки.

Далее был проведён этап проектирования IP-ядра, включающий разработку его 
структуры и алгоритма работы. В третьем и четвёртом разделах пояснительной 
записки подробно описаны процесс проектирования и обучения, особенности 
аппаратной реализации и взаимодействие с внешними устройствами.

Для проверки работоспособности IP-ядра была выполнена его программная модель и 
симуляция. Реализация и тестирование работы модели на тестовом наборе данных 
MNIST были выполнены в среде Python и Vivado, с последующей верификацией на 
FPGA.\@ Эти этапы освещены в пятом разделе пояснительной записки.

Следующим этапом является технико-экономическое обоснование разработки IP-ядра, 
включающее анализ затрат на реализацию и оценку преимуществ аппаратного 
ускорения по сравнению с программными методами. Результаты данного этапа 
приведены в шестом разделе пояснительной записки.

Завершающим этапом стала оценка производительности разработанного IP-ядра, 
включая анализ быстродействия, потребления ресурсов и точности распознавания. 
Также было проведено сравнение с существующими решениями, что позволило выявить 
преимущества и возможные направления оптимизации. Результаты данного анализа 
представлены в седьмом разделе пояснительной записки.

Отчет о проверке на заимствования представлен в приложении А.