\clearpage % Новый лист для списка литературы
\renewcommand{\refname}{\begin{center}СПИСОК ИСПОЛЬЗОВАННЫХ ИСТОЧНИКОВ\end{center}\vspace*{-2.0em}} % Установка заголовка списка

\makeatletter
\renewcommand{\@biblabel}[1]{[#1]\hfill} % Выравнивание номера ссылки
\renewenvironment{thebibliography}[1]
     {\section*{\refname}%
      \@mkboth{\MakeUppercase\refname}{\MakeUppercase\refname}%
      \begin{list}{\@biblabel{\arabic{enumi}}}{%
          \setlength{\leftmargin}{0pt} % Убираем отступы
          \setlength{\labelwidth}{1em} % Убираем ширину метки
          \setlength{\itemindent}{1.25cm} % Убираем отступы на первой строке
          \setlength{\itemsep}{0.15em} % Убираем отступы между элементами
          \setlength{\parsep}{0pt} % Убираем дополнительный вертикальный отступ
          \usecounter{enumi}%
          \let\p@enumi\@empty
          \renewcommand{\theenumi}{\arabic{enumi}}}%
      \sloppy
      \clubpenalty4000
      \widowpenalty4000%
      \sfcode`\.\@m}
     {\end{list}}
\makeatother

\addcontentsline{toc}{section}{\hspace*{-1em}Список использованных источников}\par

\begin{thebibliography}{99}
    \setlength{\itemindent}{1.95cm} % Убираем отступы для первой строки
    \setlength{\leftmargin}{1.25cm} % Обнуляем отступы для правильного форматирования
    \setlength{\labelsep}{2mm} % Убираем стандартный отступ после номера

    \raggedright% Выравнивание по левому краю (без отступов)
    \justifying
    \bibitem{ITS2024} FPGA реализация нейронной сети прямого распространения для распознавания рукописных чисел / Е. А. Кривальцевич, М. И. Вашкевич // Информационные технологии и системы 2024 (ИТС 2024) = Information Technologies and Systems 2024 (ITS 2024) : материалы международной научной конференции, Минск, 20 ноября 2024 г. / Белорусский государственный университет информатики и радиоэлектроники; редкол. : Л. Ю. Шилин [и др.]. – Минск, 2024. – С. 85–86.
    
    \bibitem{Doc} Кривальцевич Е.А., Вашкевич М.И. Аппаратная реализация нейронной сети прямого распространения для распознавания рукописных цифр на базе FPGA. Доклады БГУИР. 20**; **(*): ***-***.
    
    \bibitem{CNN} Bouvrie J. Notes on convolutional neural networks. – 2006.
    
    \bibitem{Giardino} Giardino D. et al. FPGA implementation of hand-written number recognition based on CNN // International Journal on Advanced Science, Engineering and Information Technology. – 2019. – Т. 9. – No. 1. – P. 167-171. 
    
    \bibitem{Varsamopoulos} Varsamopoulos, S. (2019). Neural Network based Decoders for the Surface Code. [Dissertation (TU Delft), Delft University of Technology]. https://doi.org/10.4233/uuid:dc73e1ff-0496-459a-986f-de37f7f250c9
    
    \bibitem{Agrawal} Agrawal V., Jagtap J., Kantipudi M. V. V. P. Decoded-ViT: A Vision Transformer Framework for Handwritten Digit String Recognition //Revue d'Intelligence Artificielle. – 2024. – Т. 38. – №. 2. – С. 523.
    
    \bibitem{Performance} Agrawal V. et al. Performance analysis of hybrid deep learning framework using a vision transformer and convolutional neural network for handwritten digit recognition //MethodsX. – 2024. – Т. 12. – С. 102554.
    
    \bibitem{YADRO} YADRO [Электронный ресурс] – Электронные данные – Режим доступа: https://itglobal.com/ru-by/solutions/tech-partners/yadro/

    \bibitem{LST2D} Maxim Vashkevich and Egor Krivalcevich Learned 2D separable transform: building block for designing compact and efficient neural network for image recognition.
    
    \bibitem{MNIST} MNIST Handwritten Numbers database [Electronic resource] – Electronic data – Access mode: https://yann.lecun.com/exdb/mnist/Mittal

    \bibitem{vivado_axi} Xilinx, {AXI Reference Guide} [Электронный ресурс] – Электронные данные – Режим доступа: https://docs.amd.com/v/u/en-US/ug1037-vivado-axi-reference-guide, 2017.

    \bibitem{chu_fpga} Pong P. Chu, {FPGA Prototyping by Verilog Examples} [Электронный ресурс] – Электронные данные – Режим доступа: https://faculty.kfupm.edu.sa/COE/aimane/coe405/FPGA\%20examples.pdf, 2018.

    \bibitem{Tokheim2003} Tokheim, R.L. Digital Electronics: Principles and Applications. – McGraw-Hill, 2003. – 480 p.

    \bibitem{Katz1994} Katz, R.H. Contemporary Logic Design. – Benjamin-Cummings, 1994. – 700 p.

    \bibitem{XilinxUG} Xilinx Inc., \textit{Vivado Design Suite User Guide}, UG892, 2021.  

    \bibitem{TANH}  L. D. Medus, T. Iakymchuk, J. V. Frances-Villora, M. Bataller-
Mompean, and A. Rosado-Munoz, “A novel systolic parallel hardware
architecture for the FPGA acceleration of feedforward neural networks,”
IEEE Access, vol. 7, pp. 76 084–76 103, 2019.


\end{thebibliography}
