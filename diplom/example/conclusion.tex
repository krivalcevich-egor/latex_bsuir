%------------------------------------------------------------------------------
% \Conclusion
%------------------------------------------------------------------------------
\section*{\vspace*{-\baselineskip}\begin{center}ЗАКЛЮЧЕНИЕ\end{center}\vspace*{-\baselineskip*2}}\par
\addcontentsline{toc}{section}{\hspace*{-1em}Заключение}\par

%------------------------------------------------------------------------------
% Text
%------------------------------------------------------------------------------
\hspace*{12.5 mm}В рамках преддипломной практики была 
разработана структурная схема IP-блока нейронной сети прямого
распространения для распознавания рукописных цифр.
Проведено исследование существующих аналогичных
разработок, выявлены их положительные и отрицательные стороны. Был
произведен анализ технического задания, разработана функциональная
спецификация системы. Также была обучена нейронная сеть, в соответствии
с выбранной архитектурой. Написана эталонная модель на 
языке python с использованием библиотеки fixpoint, а также
произведено сравнение работы модели с плавающей и 
фиксированной запятой.

   