\newcommand{\studentName}{Кривальцевичу Егору Александровичу}
\newcommand{\gr}{150701}

\thispagestyle{empty}
\clearpage
\thispagestyle{empty}

\fontsize{12pt}{12pt}\selectfont
\begin{center}
    \renewcommand{\arraystretch}{0.4} % Меняем межстрочный интервал

        \begin{tabular}{c}
            {\fontsize{11}{14}\selectfont Министерство образования Республики Беларусь} \\ \\
            {\fontsize{11}{0}\selectfont Учреждение образования} \\
            {\fontsize{10}{8}\selectfont БЕЛОРУССКИЙ ГОСУДАРСТВЕННЫЙ УНИВЕРСИТЕТ ИНФОРМАТИКИ И РАДИОЭЛЕКТРОНИКИ}
        \end{tabular}    
   
    \small
    \begin{flushleft}
        Кафедра \hspace{2cm}\underline{\hspace{2cm}Электронных вычислительных средств\hspace{2cm}} \\
    \end{flushleft}
    
    \begin{flushright}
        \text{УТВЕРЖДАЮ\hspace{2.7cm}}\\
        \text{Зав. кафедрой ЭВС \hspace{1.8cm}}\\
        \underline{\hspace{3cm}} 
        И.С. Азаров \\[1em]
        \textit{\underline{«\hspace{0.2cm}06\hspace{0.2cm}» \hspace{0.7cm}марта \hspace{0.7cm}2025 г.}}
    \end{flushright}
    \normalfont

    \vspace{1.5em}
    
    \textbf{ЗАДАНИЕ}\\
    \textbf{по дипломному проекту студента}
    
    \fontsize{12pt}{12pt}\selectfont
    \noindent
    \makebox[\textwidth]{
    Обучающемуся \hspace{1cm} \underline{\textit{\hspace{3cm} \studentName \hspace{2.6cm}}}
    }

    \fontsize{10}{0}\hspace{3.0cm}\fontsize{10pt}{10pt}\selectfont(фамилия, собственное имя, отчество(если таковое имеется))
\end{center}
\fontsize{12pt}{12pt}\selectfont
\small

\noindent
Курс \hspace{1cm}  \underline{\hspace{1cm} 4\hspace{1cm}}\hspace{0.5cm} Учебная группа \hspace{0.5cm}\underline{\hspace{3.5cm} \gr \hspace{3.5cm}} \\
Специальность \hspace{2cm}  \underline{\hspace{1cm} 1-40 02 02 <<Электронные вычислительные средства>>\hspace{1.3cm}}\\
\underline{{Тема дипломного проекта:} \textit{IP-ядро нейронной сети прямого распространения для}\hspace{2cm}}\\
\underline{\textit{распознавания рукописных цифр}\hspace{9.0cm}}\\[0.5em]
\underline{\hspace{16.7cm}}\\[0.5em]
\text{утверждена приказом по университету от \hspace{2cm} \underline{«\textit{ 31 }» \hspace{0.35cm}\textit{{ марта }} \hspace{0.35cm}2025 }г.\hspace{0.55cm} № \underline{ \hspace{0.1cm}878-с \hspace{0.1cm}}}

\vspace{1em}

\noindent
\text{\textbf{2. Срок сдачи студентом законченного проекта} \underline{\hspace{2.2cm}15 июня 2025 года\hspace{2.2cm}}}

\vspace{1em}

\noindent
{\textbf{3. Исходные данные к проекту}}

\noindent
{\textit{3.1 Назначение разработки:} система предназначена для распознавания рукописных цифр.}\\[0.2em]
{\textit{3.2 Технические характеристики:}}\\[-0.3em]
{\textit{– размер изображения: 28x28;}}\\[-0.3em]
{\textit{– цвет изображения: в оттенках серого;}}\\[-0.3em]
{\textit{– функция активации: softmax.}}\\[-0.3em]

\noindent
\textbf{4. Содержание пояснительной записки (перечень подлежащих разработке вопросов)}

\noindent
\textit{4.1 Введение.
4.2 Обзор существующих нейронных сетей для распознавания рукописных цифр.
4.3 Анализ технического задания.
4.4 Разработка структуры IP-блока нейронной сети для распознавания рукописных цифр.
4.5 Разработка алгоритма работы IP-блока нейронной сети для распознавания рукописных цифр.
4.6 Аппаратная реализация IP-блока нейронной сети для распознавания рукописных цифр.
4.7 Технико-экономическое обоснование разработки IP-блока нейронной сети для распознавания рукописных цифр.
4.8 Анализ результатов тестирования IP-блока нейронной сети для распознавания рукописных цифр.
4.9 Заключение.
4.10 Список используемых источников.}

% \vspace{5em}
\vspace{1em}

\noindent
\textbf{5. Перечень графического материала (с указанием названия чертежей и их количества в перечне на формат A1)}

\vspace{0.3em}
\noindent
5.1 Структура IP-блока нейронной сети для распознавания рукописных цифр – 1 лист формата A1 (плакат).\\
\clearpage
\thispagestyle{empty}
\noindent
5.2 Граф схема алгоритма IP-блока нейронной сети для распознавания рукописных цифр – 1 лист формата A1.\\
5.3 Схема LST обработки изображения в IP-блоке нейронной сети для распознавания рукописных цифр – 1 лист формата A1.\\
5.4 Аппаратные затраты на реализацию IP-блока нейронной сети для распознавания рукописных цифр – 1 лист формата A1 (плакат).\\
5.5 Функциональная схема IP-блока нейронной сети для распознавания рукописных цифр – 1 лист формата A1.\\
5.6 Экспериментальное исследование IP-блока нейронной сети для распознавания рукописных цифр – 1 лист формата A1 (плакат).\\

\noindent
\textbf{6. Содержание задания по технико-экономическому обоснованию}

\noindent
\uline{\textit{6.1. Характеристика программно-аппаратного комплекса.\hspace{5.8cm}}}\\[0.3em]
\textit{\uline{6.2. Расчет экономического эффекта от производства программно-аппаратного комплекса.\hspace{14.5cm}}}\\[0.3em]
\textit{\uline{6.3. Расчет инвестиций в производство программно-аппаратного комплекса.\hspace{2.6cm}}}\\[0.3em]
\textit{\uline{6.4. Расчет показателей экономической эффективности инвестиций в производство программно-аппаратного комплекса.\hspace{9.75cm}}}

\vspace{1em}

\noindent
Задание выдал: \underline{\hspace{5cm}} / И. В. Смирнов /

\vspace{2em}

\begin{center}
    \text{КАЛЕНДАРНЫЙ ПЛАН}
\vspace{-0.4em}
\begin{longtable}{|p{8cm}|>{\centering\arraybackslash}p{1.5cm}|>{\centering\arraybackslash}p{3.5cm}|>{\centering\arraybackslash}p{2cm}|}
    \hline
    \centering{Наименование этапов дипломного проекта (работы)} & \centering{Объём этапа \%} & \centering{Срок выполнения этапа} & {Примеча-ние} \\
    \hline
    I этап – п.4.1 – 4.4, 5.1 – 5.3 & 60 & 20.04.22 & \\ \hline
    II этап – п.4.5 – 4.5.4 & 20 & 27.04.22 & \\ \hline
    III этап – п.4.6 – 4.8, п.5.5, 5.6 & 20 & 11.05.22 & \\ \hline
    Нормоконтроль & & 16.05.22 – 20.05.22 & \\ \hline
    Рабочая комиссия & & 23.05.22 – 27.05.22 & \\ \hline
    Рецензирование & & 30.05.22 – 10.06.22 & \\ \hline
    Защита & & 15.06.22 – 30.06.22 (в соответствии с графиком заседаний ГЭК) &  \\ \hline
\end{longtable}
\end{center}

\vspace{2em}

\noindent
Дата выдачи задания \underline{«31» марта 2025 г.} \hfill Руководитель \underline{\hspace{3cm}} / М. И. Вашкевич /

\vspace{2em}

\noindent
Задание принято к исполнению \underline{\hspace{3cm}} / Е. А. Кривальцевич /
\normalfont
\clearpage
