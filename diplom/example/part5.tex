%------------------------------------------------------------------------------
% \PART 5
%------------------------------------------------------------------------------
\chapter[Аппаратная реализация IP-блока нейронной сети для распознавания рукописных цифр]
{АППАРАТНАЯ РЕАЛИЗАЦИЯ IP-БЛОКА НЕЙРОННОЙ СЕТИ ДЛЯ РАСПОЗНАВАНИЯ РУКОПИСНЫХ ЦИФР}

%------------------------------------------------------------------------------
% \PART 5.1 Text
%------------------------------------------------------------------------------
\section{Описание пакета Xilinx Vivado}
\hspace*{12.5 mm}Xilinx Vivado – это современный программный пакет для 
проектирования цифровых схем на базе программируемых логических интегральных 
схем (ПЛИС, FPGA), разработанный компанией Xilinx. Данный программный комплекс 
является преемником Xilinx ISE и значительно превосходит его по возможностям и 
производительности. Vivado предназначен для автоматизированного проектирования 
цифровых схем, начиная с этапа описания логики на языках описания аппаратуры 
(HDL) и заканчивая загрузкой сгенерированного битового потока в 
устройство\cite{XilinxUG}.  

Одним из ключевых преимуществ Vivado является использование более 
производительных алгоритмов синтеза и оптимизации, что позволяет существенно 
ускорить процесс разработки. В отличие от традиционных средств проектирования, 
в Vivado реализована поддержка параллельных вычислений, что делает его особенно
эффективным при работе с большими проектами.  

В состав пакета Vivado входят несколько инструментов, среди 
которых\cite{XilinxUG}:  

  1 {Vivado IDE} – интегрированная среда разработки, предоставляющая удобный 
графический интерфейс для управления проектами, анализа схемотехнической логики
и моделирования.

  2 {Vivado Synthesis} – инструмент синтеза логики, который преобразует код HDL
в оптимизированное представление на уровне вентилей.

  3 {Vivado Implementation} – модуль размещения, трассировки и оптимизации 
логики на кристалле FPGA.\@

  4 {Vivado Simulator} – встроенный инструмент для   моделирования цифровых 
схем на уровне поведенческой, функциональной и временной верификации.

  5 {Vivado IP Integrator} – инструмент для работы с IP-ядрами и объединения их
в сложные системы на кристалле.

  6 {Vivado High-Level Synthesis (HLS)} – средство высокоуровневого синтеза, 
позволяющее описывать аппаратные алгоритмы на языках C/C++ и автоматически 
преобразовывать их в оптимизированный RTL-код.

Основные возможности Vivado:  

  1 Поддержка языков описания аппаратуры VHDL и Verilog. Vivado позволяет 
разрабатывать схемы на двух основных языках HDL, а также поддерживает 
SystemVerilog для верификации цифровых проектов. Это дает инженерам широкие 
возможности для разработки и тестирования аппаратуры.

   2 Автоматизированный синтез и размещение логики на FPGA.\@ Vivado включает 
мощный механизм синтеза, который автоматически преобразует HDL-код в 
оптимизированную сетевую схему, выполняет ее размещение на кристалле FPGA и 
прокладывает соединения между логическими блоками.

  3 Встроенные инструменты для моделирования и отладки. 

  4 Генерация IP-ядер и использование встроенных библиотек компонентов. 


Для реализации нейронной сети Vivado используется на нескольких ключевых этапах
разработки. Во-первых, он применяется для генерации аппаратной логики, 
необходимой для выполнения расчетов внутри FPGA.\@ Во-вторых, с его помощью 
производится компиляция проекта и синтез аппаратной схемы. В-третьих, Vivado 
используется для загрузки сгенерированного битового потока в ПЛИС и отладки 
работы модели.  

%------------------------------------------------------------------------------
% \PART 5.2 Text
%------------------------------------------------------------------------------
\section{Описание функциональных блоков}
\hspace*{12.5 mm}Аппаратная реализация нейронной сети включает несколько 
ключевых функциональных блоков:

  1 Блок обработки пикселя.

  2 Блок вычисления тангенса.

  3 Блок вычисления softmax.

  4 Блок памяти изображения.

  5 Блок управляющего автомата.

  6 Блок умножения.

Каждый из этих блоков реализуется в виде аппаратного модуля на HDL.\@

В блоке обработки пикселя осуществляется установка параметра смещения для 
подготовки к вычислению слоя, а также выбираетсясоответствующий данному этапу 
весовой коэффициент из памяти.

В блоке вычисления тангенса осуществляется аппроксимированный расчет функции 
активации тангенс гиперболический для LST-1 слоя.

Прямая реализация вычисления тангенса будет занимать большое количество 
вычислительных ресурсов. Поэтому для удобства реализации используется 
аппроксимация в соответствии с формулой\cite{TANH}:

\begin{equation}
    \tanh{(x)}\approx F(x) = 
      \begin{cases}
      \mathrm{sign}(x),         & \mathrm{if}\;\;  |x|>2, \\
      (1 + \frac{x}{4})\cdot x, & \mathrm{if}\;\; -2<x<0, \\
      (1 - \frac{x}{4})\cdot x, & \mathrm{if}\;\;  0<x<2. \\
      \end{cases}
\end{equation}

В блоке вычисления softmax происходит сравнение входных данных и определение 
наиболее вероятного класса.

В блоке памяти изображения храниться результат обработки на каждом этапе. От 
приемки входных данных, до выхода LST-1 слоя.

Блок управляющего автомата осуществляет управление и синхронизацию между всеми 
блоками устройства и обрабатывает входные управляющие сигналы.

Блок умножения состоит из матричного умножителя, который описан с целью анализа
аппаратных затрат и точности при изменении разрядности весовых коэффициентов.

Реализация данных блоков на языке SystemVerilog представлена в приложении З.

%------------------------------------------------------------------------------
% \PART 5.3 Text
%------------------------------------------------------------------------------
\section{Описание интерфейса нейронной сети}

\hspace*{12.5 mm}Как было показано ранее на рисунке 4.5 нейронная сеть состоит
из пяти обязательных входных и двух выходных сигналов. 
Взаимодействие процессорной системы и программируемой логики 
осуществляется через интерфейс AXI4-lite. Таким образом 
необходимым условием, выдвигаемым к IP-блоку будет поддержка 
интерфейса AXI4-lite. Для упрощения формирования IP-блока 
будем использовать связку AXI-uP. uP интерфейс напрямую 
взаимодействует с модулем верхнего уровня. Далее данные 
передаются в AXI-slave модуль, который и подключается к 
процессорной системе.

%------------------------------------------------------------------------------
% \PART 5.4 Text
%------------------------------------------------------------------------------
\section{Описание процесса создания блочной диаграммы}
\hspace*{12.5 mm}Создадим блочную диаграмму, что показано на
рисунке~\ref{fig:creat-bd}.

\insertfigure{creat-bd}{pic/create_bd.jpg}{Создание блочной диаграммы}{6cm}

Для создания блочной диаграммы необходимо создать IP-блок нейронной сети. Для 
этого сначала выполним добавление всех файлов в IP-блок, что представлено на 
рисунке~\ref{fig:add-files}

\insertfigure{add-files}{pic/add_files.jpg}{Добавление файлов}{13cm}

Далее необходимо добавить на блочную диаграмму процессорную систему, а затем 
IP-блок. Также необходимыми элементами является AXI-Interconnect и системный 
Processor System Reset. AXI-Interconnect используется для связи процессорной 
системы с несколькими IP-блоками. Он выполняет маршрутизацию транзакций между 
мастером (процессором) и ведомыми устройствами. Блок Processor System Reset 
управляет сигналами сброса в системе. Он синхронизирует сброс процессора, 
периферийных устройств и пользовательских модулей. Полученная блочная диаграмма
показана на рисунке~\ref{fig:bd}.

\insertfigure{bd}{pic/bd}{Результирующий вид блочной диаграммы}{16cm}

%------------------------------------------------------------------------------
% \PART 5.5 Text
%------------------------------------------------------------------------------
\section{Результаты синтеза и симуляции}
\hspace*{12.5 mm}Для проверки работоспособности написанного устройства был 
составлен тест, в котором формируются управляющие воздействия и изображение. 
Работа устройства начинается по сигналу start. Для тестирования было взято два 
изображения, что показано на рисунке~\ref{fig:num}.

\insertfigurescustom{pic/num1.png}{pic/num4.png}{5cm}{Тестируемые изображения}{num}{sidebyside}

Тест представлен в приложении З. Для проверки корректности работы нейронной 
сети необходимо сравнить выход результирующего полносвязного слоя и нейронной 
сети. На рисунках~\ref{fig:n1} и~\ref{fig:n4} представлены временные диаграммы 
описанной на SystemVerilog нейронной сети и выход эталонной нейронной сети на 
языке Python.

\insertfigurescustom{pic/t1.jpg}{pic/num1_et.png}{16cm}{Временная диаграмма нейронной сети и результат работы эталона на первом изображении}{n1}{stacked}
\insertfigurescustom{pic/t2.jpg}{pic/num4_et.png}{16cm}{Временная диаграмма нейронной сети и результат работы эталона на втором изображении}{n4}{stacked}

В результате мы наблюдаем полное соответствие работы эталонной и реализованной 
моделей. Таким образом данную реализацию можно считать корректной. 

Следующим не мало важным аспектом является аппаратные затраты на реализацию 
данной нейронной сети. Был произведен синтез данного RTL-кода в двух вариациях:

  1 В качестве умножителей используются блоки DSP48. Блок DSP48 представляет 
собой специализированный программируемый вычислительный блок, встроенный в FPGA, 
предназначенный для выполнения арифметических операций, таких как умножение, 
сложение и накопление сумм. Архитектура DSP48 оптимизирована для эффективного 
выполнения операций цифровой обработки сигналов и интенсивных вычислений.

  2 Во второй вариации в качестве умножителей используются описанные матричные 
умножители, что удобно для оценки аппаратных затрат в зависимости от 
используемой разрядности данных.

Аппаратные затраты представлены на рисунке~\ref{fig:hw}.

\insertfigurescustom{pic/hw1.jpg}{pic/hw2.jpg}{8cm}{Аппаратные затраты на DSP48 и матричном умножителе}{hw}{sidebyside}

