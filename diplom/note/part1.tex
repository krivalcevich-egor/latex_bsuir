%------------------------------------------------------------------------------
% \PART 1
%------------------------------------------------------------------------------
\chapter[Обзор существующих ...]
{ОБЗОР СУЩЕСТВУЮЩИХ ...}

%------------------------------------------------------------------------------
% \PART 1.1 Text
%------------------------------------------------------------------------------
\section{Существующие ...}\par
\hspace*{12.5 mm}Распознавание 

%------------------------------------------------------------------------------
% \PART 1.2 Text
%------------------------------------------------------------------------------
\section{Устройство 1}\par
\hspace*{12.5 mm}Полносвязная нейронная сеть 

%------------------------------------------------------------------------------
% \PART 1.2.1 Text
%------------------------------------------------------------------------------
\subsection{Структура устройства 1}\par
\hspace*{12.5 mm}Полносвязная 

Пример использования ссылки на рисунок.
% Общая структура полносвязных нейронных сетей показана на 
% рисунке~\ref{fig:fcnn}.

Пример формулы:

\begin{equation}
    {h} = f({W} {x} + {b})
\end{equation}

\noindentгде: ${x}$ — входной вектор размерности $d$,
              ${W}$ — матрица весов размерности $n \times d$,
              ${b}$ — вектор смещений размерности $n$,
              $f(\cdot)$ — функция активации (например, ReLU, сигмоида),
              ${h}$ — выходной вектор нейронов текущего слоя.

Пример вставки изображения.              
% \insertfigure{fcnn}{pic/fcnn.png}{Общая структура полносвязных нейронных сетей}{13.5cm}

\subsection{Проблемы устройства 1}
\hspace*{12.5 mm}Обычные 

%------------------------------------------------------------------------------
% \PART 1.3 Text
%------------------------------------------------------------------------------
\section{Устройство 2}
\hspace*{12.5 mm}
Сверточные нейронные сети

%------------------------------------------------------------------------------
% \PART 1.3.1 Text
%------------------------------------------------------------------------------
\subsection{Структура устройства 2}
\hspace*{12.5 mm}Стандартная архитектура 

%------------------------------------------------------------------------------
% \PART 1.3.2 Text
%------------------------------------------------------------------------------
\subsection{Блок ...}
\hspace*{12.5 mm}Основная 

\subsection{Проблемы устройства 2}
\hspace*{12.5 mm}Обычные 

%------------------------------------------------------------------------------
% \PART 1.4.3 Text
%------------------------------------------------------------------------------
\subsection{Применение устройства 2}
\hspace*{12.5 mm}Устройство
