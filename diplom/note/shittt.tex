\documentclass[a4paper,14pt]{extreport} % Основной класс документа — отчёт, ГОСТ-совместимый, 14 pt шрифт

\usepackage{bsuir2025}                  % Подключаем наш собственный стиль оформления

\begin{document}

% -----------------------------------------------------------------------------
% ТИТУЛЬНЫЕ ЛИСТЫ
% -----------------------------------------------------------------------------
\clearpage
\thispagestyle{empty}            % Без номера страницы
% title.tex
\begin{center}
  Министерство образования Республики Беларусь \\
  \bigskip
  Учреждение образования \\
  "<БЕЛОРУССКИЙ ГОСУДАРСТВЕННЫЙ УНИВЕРСИТЕТ ИНФОРМАТИКИ И РАДИОЭЛЕКТРОНИКИ"> \\
  \bigskip \bigskip
  Факультет компьютерных сетей и систем \\
  Кафедра электронных вычислительных средств \\
  \bigskip \bigskip \bigskip \bigskip \bigskip \bigskip \bigskip \bigskip \bigskip \bigskip \bigskip \bigskip
  Отчет по лабораторной работе № 1 \\
  \bf{"<Тема лабораторной работы">}\normalfont \\
  \bigskip \bigskip \bigskip \bigskip \bigskip \bigskip \bigskip \bigskip \bigskip \bigskip \bigskip \bigskip
  \begin{table}[ht]
    \begin{tabular}{p{7cm} p{1cm} p{7cm}}
        Выполнили студенты гр. X5070X \par XXX X.X. \par XXX X.X.\par XXX X.X.& \hfill & \hspace*{30 mm}Проверил \par \hspace*{29 mm} XXX X.X.
    \end{tabular}
  \end{table}
  \vfill \vfill
  \centerline{Минск 2025}
\end{center}
\newpage
                  % Основной титульный лист

\clearpage
\thispagestyle{empty}
% Данные документа

\begin{center}
  \text{Министерство образования Республики Беларусь}\\
  \text{Учреждение образования}\\
  \text{<<Белорусский государственный университет}\\
  \text{информатики и радиоэлектроники>>}\\[5em]

  \small
  \begin{flushright}
    Номер зачетной книжки\underline{\hspace*{3.6cm}}\\
    Преддипломная практика зачтена с оценкой\\
    \underline{\hspace*{1.3cm}} (\underline{\hspace*{3cm}})\hspace{3.2cm}\hspace*{2.1cm}\\
    \fontsize{8pt}{8pt}\selectfont
    (цифрой) \hspace{1.2cm} (прописью)\hspace{4.1cm}\hspace*{5.9cm}\\
    \normalfont
    \fontsize{12pt}{12pt}\selectfont
    \underline{\hspace*{5.4cm}}\hspace*{2.5cm}\hspace*{2.3cm}\\
    \fontsize{8pt}{8pt}\selectfont
    \hspace{2.5cm}(подпись руководителя практики от БГУИР)\hspace{2.6cm}\hspace*{5.4cm}\\
    \normalfont
    \fontsize{12pt}{12pt}\selectfont
    \underline{\hspace*{1cm}}.\underline{\hspace*{1cm}}.2025\hspace*{4.8cm}
  \end{flushright}
  \normalfont

  \fontsize{14pt}{14pt}\selectfont
  \vfill

  \textbf{ОТЧЕТ}\\
  \textbf{по преддипломной практике}\\
  \normalfont
  \text{Место прохождения практики: ЗАО <<Инженерный центр ЯДРО>>}\\
  \text{Сроки прохождения практики: с 10.02.2025 по 23.03.2005}\\[3em]


\begin{flushleft}
  \begin{tabular}{ p{0.4\textwidth} p{0.1\textwidth} p{0.5\textwidth} }
    Руководитель практики от & & Студент группы 150701\\
    предприятия: & & \underline{\hspace*{3cm}}Е. А. Кривальцевич\\[-1em]
    & & \vspace{0mm} \fontsize{8pt}{8pt}\selectfont\hspace*{0.35cm}(подпись студента)\normalsize\\
    \vspace{-7mm}\underline{\hspace*{3cm}}П. Н. Габер & & Руководитель практики от БГУИР \\
    \vspace{-9mm}  \fontsize{8pt}{8pt}\selectfont(подпись руководителя)\normalsize & & \\
    \vspace{-11mm} \fontsize{8pt}{8pt}\selectfont М.П. \normalsize & & \vspace{-9mm}\makebox[0.43\textwidth][s]{Вашкевич М.И.  -- профессор}
    \\
    & & \vspace{-9mm} кафедры ЭВС\\
  \end{tabular}
\end{flushleft}

  \vfill
  {\city~\targetYear}
\end{center}

          % Титульный лист практики (если есть)

% -----------------------------------------------------------------------------
% ТЕХНИЧЕСКОЕ ЗАДАНИЕ (скан, PDF)
% -----------------------------------------------------------------------------
\clearpage
\thispagestyle{empty}
\includepdf[pages=1, pagecommand={}, fitpaper, offset=0 0]{TZ.pdf} % Вставляем первую страницу ТЗ (PDF)

\clearpage
\thispagestyle{empty} 
\includepdf[pages=2, pagecommand={}, fitpaper, offset=0 0]{TZ.pdf} % Вставляем вторую страницу ТЗ

% -----------------------------------------------------------------------------
% ОСНОВНОЙ ДОКУМЕНТ
% -----------------------------------------------------------------------------

\normalfont                         % Устанавливаем стандартный шрифт
\newpage                            % Устанавливаем новую страницу
\setcounter{page}{4}                % Устанавливаем нумерацию с 4 страницы

% Колонтитулы
\pagestyle{fancy}
\fancyhf{}                          % Очищаем все колонтитулы
\fancyfoot[R]{\thepage}             % Номер страницы справа внизу
\renewcommand{\headrulewidth}{0pt}  % Без линии сверху
\renewcommand{\footrulewidth}{0pt}  % Без линии снизу

% Для "plain"-страниц (например, после \chapter)
\fancypagestyle{plain}{%
  \fancyhf{}
  \fancyfoot[R]{\thepage}
  \renewcommand{\headrulewidth}{0pt}
  \renewcommand{\footrulewidth}{0pt}
}

% Настройки шрифта
\normalfont\fontsize{14pt}{14pt}\selectfont

% -----------------------------------------------------------------------------
% ОГЛАВЛЕНИЕ
% -----------------------------------------------------------------------------
\let\oldtableofcontents=\tableofcontents
\renewcommand{\tableofcontents}{\begingroup\parskip=0pt \oldtableofcontents\endgroup}

\noindent\tableofcontents
\addtocontents{toc}{\protect\thispagestyle{fancy}}  % Устанавливаем стиль fancy в оглавлении

% Разрешаем переносы после оглавления
\hyphenpenalty=50
\tolerance=2000

% -----------------------------------------------------------------------------
% ВКЛЮЧЕНИЕ ОСНОВНЫХ РАЗДЕЛОВ
% (Каждая секция — отдельный .tex-файл)
% -----------------------------------------------------------------------------
%------------------------------------------------------------------------------
% \Introduction
%------------------------------------------------------------------------------
\chapter*{\vspace*{-\baselineskip}\begin{center}ВВЕДЕНИЕ\end{center}\vspace*{-\baselineskip*2}}
\addcontentsline{toc}{chapter}{\hspace*{-1em}Введение}\par

%------------------------------------------------------------------------------
% Text
%------------------------------------------------------------------------------
\hspace*{12.5 mm}Распознавание рукописных цифр является одной из ключевых 
задач в области обработки изображений и компьютерного зрения. Оно находит 
применение в различных сферах, таких как банковские системы, почтовые службы, 
идентификация документов и автоматизация процессов ввода данных. Современные 
технологии позволяют решать эту задачу с высокой точностью благодаря 
использованию методов машинного обучения и нейронных сетей.

В последние годы значительный интерес вызывает аппаратная реализация нейронных 
сетей, поскольку программные решения, работающие на центральных процессорах 
(CPU) и графических процессорах (GPU), не всегда обеспечивают требуемую 
скорость обработки и энергоэффективность. В связи с этим использование 
специализированных аппаратных ускорителей на базе FPGA (Field-Programmable 
Gate Array) или ASIC (Application-Specific Integrated Circuit) становится 
актуальной задачей. Аппаратная реализация нейронной сети в виде IP-ядра 
позволяет значительно повысить производительность и снизить задержки обработки 
данных, что особенно важно для встроенных систем.

При реализации нейронных сетей на базе CPU и GPU как правило используются 
стандартизированные типы данных (чаще всего числа с плавающей запятой одинарной
точности, реже целочисленные типы). При реализации на базе FPGA появляется 
возможность использовать для представления параметров нейронных сетей типов 
данных, обеспечивающих различную точность. Причем выбор точности представления 
напрямую будет влиять на аппаратные затраты\cite{ITS2024}. 

Целью данного дипломного проекта является разработка IP-ядра нейронной сети 
прямого распространения для распознавания рукописных цифр. В данном проекте 
используется обученное двумерное разделяемое преобразование (LST2D), которое 
рассматривается как новый тип слоя нейронной сети. Он следует концепции 
распределения веса. Проектирование IP-ядра нейронной сети прямого 
распространения производилось в несколько этапов, отраженных в структуре 
пояснительной записки данного дипломного проекта.

Первоначально был проведён анализ существующих методов и архитектур нейронных 
сетей, применяемых для задач распознавания рукописных цифр. Рассматривались 
различные подходы, включая классические многослойные персептроны, свёрточные 
и рекуррентные нейронные сети. Итоги данного этапа изложены в первом разделе 
пояснительной записки.

На следующем этапе выполнен анализ технического задания, в результате которого 
была выбрана архитектура нейронной сети, наиболее подходящая для аппаратной 
реализации. Основные критерии выбора включают компактность модели и 
вычислительную эффективность. Результаты этого анализа представлены во втором 
разделе пояснительной записки.

Далее был проведён этап проектирования IP-ядра, включающий разработку его 
структуры и алгоритма работы. В третьем и четвёртом разделах пояснительной 
записки подробно описаны процесс проектирования и обучения, особенности 
аппаратной реализации и взаимодействие с внешними устройствами.

Для проверки работоспособности IP-ядра была выполнена его программная модель и 
симуляция. Реализация и тестирование работы модели на тестовом наборе данных 
MNIST были выполнены в среде Python и Vivado, с последующей верификацией на 
FPGA.\@ Эти этапы освещены в пятом разделе пояснительной записки.

Следующим этапом является технико-экономическое обоснование разработки IP-ядра, 
включающее анализ затрат на реализацию и оценку преимуществ аппаратного 
ускорения по сравнению с программными методами. Результаты данного этапа 
приведены в шестом разделе пояснительной записки.

Завершающим этапом стала оценка производительности разработанного IP-ядра, 
включая анализ быстродействия, потребления ресурсов и точности распознавания. 
Также было проведено сравнение с существующими решениями, что позволило выявить 
преимущества и возможные направления оптимизации. Результаты данного анализа 
представлены в седьмом разделе пояснительной записки.

Отчет о проверке на заимствования представлен в приложении А.               % Введение
%------------------------------------------------------------------------------
% \PART 1
%------------------------------------------------------------------------------
\chapter[Обзор существующих нейронных сетей для распознавания рукописных цифр]
{ОБЗОР СУЩЕСТВУЮЩИХ НЕЙРОННЫХ СЕТЕЙ ДЛЯ РАСПОЗНАВАНИЯ РУКОПИСНЫХ ЦИФР}

%------------------------------------------------------------------------------
% \PART 1.1 Text
%------------------------------------------------------------------------------
\section{Существующие нейронные сети для распознавания рукописных цифр}\par
\hspace*{12.5 mm}Распознавание рукописных цифр является одной из классических 
задач машинного обучения, для решения которой применяется широкий спектр 
нейронных сетей. В зависимости от сложности задачи, требований к точности, 
скорости обработки и аппаратных ограничений используются различные архитектуры, 
включая полносвязные сети, сверточные нейронные сети и более сложные гибридные 
модели.

Одним из первых методов распознавания рукописных цифр стали многослойные 
перцептроны (MLP), относящиеся к классу полносвязных нейронных сетей (FCNN). 
Такие сети состоят из входного слоя, нескольких скрытых слоев и выходного слоя, 
где каждый нейрон соединен со всеми нейронами предыдущего и последующего слоев. 
Несмотря на их простоту, они способны успешно решать задачу распознавания, но 
требуют большого количества параметров и вычислительных ресурсов, что делает 
их менее эффективными по сравнению с более специализированными архитектурами.

Для обработки изображений, включая рукописные цифры, более эффективными 
оказались сверточные нейронные сети (CNN). Эти сети включают сверточные слои, 
которые извлекают характерные признаки из изображения, а также слои 
субдискретизации для уменьшения размерности. CNN значительно превосходят 
полносвязные сети в точности и скорости работы, поскольку используют 
пространственные зависимости в данных и требуют меньше параметров.

Хотя рекуррентные нейронные сети (RNN) и их усовершенствованные версии, такие 
как Long Short-Term Memory Network (LSTM) и Gated Recurrent Unit Network (GRU),
чаще применяются для обработки последовательных данных, они могут 
использоваться и для распознавания рукописных цифр. В частности, они 
эффективны в случаях, когда рукописные символы анализируются в контексте 
строки, например, при распознавании рукописного текста.

Современные архитектуры, такие как Vision Transformer (ViT) и его модификации, 
предлагают альтернативный подход к обработке изображений, основанный на 
механизме самовнимания. Хотя традиционно трансформеры использовались в 
обработке естественного языка (NLP), их успешное применение в компьютерном 
зрении позволило достичь новых высот в распознавании символов и цифр.

Для повышения точности и эффективности могут применяться гибридные подходы, 
комбинирующие преимущества разных типов нейронных сетей. Например, модели, 
совмещающие CNN и LSTM, успешно применяются для распознавания рукописного 
текста, где CNN используется для выделения признаков, а LSTM — для анализа 
последовательностей.

На сегодняшний день существует множество архитектур нейронных сетей, способных 
эффективно решать задачу распознавания рукописных цифр. Выбор конкретного 
метода зависит от требований к точности, вычислительным ресурсам и целевой 
платформе. В последующих разделах будет представлен детальный анализ наиболее 
распространенных нейросетевых моделей, применяемых для данной задачи.

%------------------------------------------------------------------------------
% \PART 1.2 Text
%------------------------------------------------------------------------------
\section{Полносвязные нейронные сети}\par
\hspace*{12.5 mm}Полносвязная нейронная сеть (Fully Connected Neural Network) — 
это базовая архитектура искусственных нейронных сетей, в которой каждый нейрон 
одного слоя соединен со всеми нейронами следующего слоя. Такой тип архитектуры 
используется для различных задач, включая распознавание рукописных цифр, но 
чаще всего применяется в качестве классификатора после сверточных или 
рекуррентных слоев. Количество нейронов на каждом слое может отличаться от 
соседнихслоев.

%------------------------------------------------------------------------------
% \PART 1.2.1 Text
%------------------------------------------------------------------------------
\subsection{Структура полносвязного слоя}\par
\hspace*{12.5 mm}Полносвязная нейронная сеть состоит из следующих основных 
компонентов:

    1 \text{Входной слой} — принимает данные (например, изображение 
    $28\times28$ пикселей в задаче распознавания рукописных цифр, развёрнутое в
    вектор размером 784).

    2 \text{Скрытые слои} — выполняют нелинейные преобразования входных данных. 
    Обычно включают несколько таких слоев.

    3 \text{Выходной слой} — содержит количество нейронов, соответствующее 
    числу классов (например, 10 для цифр 0–9), и использует функцию активации, 
    например softmax. Функция активации — это математическая функция, которая 
    определяет, передаст ли нейрон сигнал дальше по сети.

Общая структура полносвязных нейронных сетей показана на 
рисунке~\ref{fig:fcnn}.

Математически данную сеть можно описать формулой для вычисления выхода нейрона 
в полносвязном слое:

\begin{equation}
    {h} = f({W} {x} + {b})
\end{equation}

\noindentгде: ${x}$ — входной вектор размерности $d$,
              ${W}$ — матрица весов размерности $n \times d$,
              ${b}$ — вектор смещений размерности $n$,
              $f(\cdot)$ — функция активации (например, ReLU, сигмоида),
              ${h}$ — выходной вектор нейронов текущего слоя.

Для многослойной архитектуры выход слоя $l$ является входом для следующего слоя, 
что можно описать уравнением:

\begin{equation}
    {h}^{(l+1)} = f({W}^{(l)} {h}^{(l)} + {b}^{(l)})
\end{equation}

\insertfigure{fcnn}{pic/fcnn.png}{Общая структура полносвязных нейронных сетей}{13.5cm}

%------------------------------------------------------------------------------
% \PART 1.2.1 Text
%------------------------------------------------------------------------------
\subsection{Функция активации ReLU}
\hspace*{12.5 mm}ReLU (Rectified Linear Unit) — одна из наиболее популярных и 
широко используемых функций активации в нейронных сетях. Она определяется 
следующим образом:

\begin{equation}
    f(x) = \max(0, x)
\end{equation}

График функции ReLU представлен на рисунке~\ref{fig:relu}.

\insertfigure{relu}{pic/ReLU.png}{График функции ReLU}{7cm}

Функция ReLU ускоряет обучение сети и устранят проблему исчезающего градиента, 
так как имеет линейное поведение для положительных значений. Сама функция 
проста в реализации, но может быть чувствительна к большим значениям градиентов
и в отрицательной области возвращает 0.

%------------------------------------------------------------------------------
% \PART 1.2.2 Text
%------------------------------------------------------------------------------
\subsection{Функция активации сигмоида}
\hspace*{12.5 mm}Сигмоида (Logistic Function) — это одна из классических 
функций активации, используемая в нейронных сетях. Её математическое 
определение:

\begin{equation}
    f(x) = \frac{1}{1 + e^{-x}}
\end{equation}\\[-9mm]

График функции сигмоида представлен на рисунке~\ref{fig:sigmoid}.

\insertfigure{sigmoid}{pic/sigmoid.png}{График функции сигмоида}{7cm}

Сигмоида подходит для моделирования вероятностей, так как имеет диапазон 
значений $ (0,1) $. Функция гладкая и дифференцируемая, что упрощает вычисление
градиента, однако выходы не центрированы вокруг нуля, что замедляет обучение.

%------------------------------------------------------------------------------
% \PART 1.2.3 Text
%------------------------------------------------------------------------------
\subsection{Функция активации Softmax}
\hspace*{12.5 mm}Функция Softmax применяется в многоклассовой классификации и 
преобразует входные значения в вероятностное распределение:

\begin{equation}
    \sigma{(z_i)} = \frac{e^{z_i}}{\sum_{j} e^{z_j}},
\end{equation}

Гистограмма функции Softmax представлен на рисунке~\ref{fig:sm}.

\insertfigure{sm}{pic/softmax.png}{Гистограмма функции Softmax}{7cm}

\noindentгде $z_i$ — входное значение для $i$-го класса, а знаменатель 
представляет собой сумму экспонент всех входных значений.

Softmax позволяет интерпретировать выходные значения сети как вероятности 
принадлежности к классам. Однако склонна к затуханию градиента при слишком 
больших входных значениях.

%------------------------------------------------------------------------------
% \PART 1.2.4 Text
%------------------------------------------------------------------------------
\subsection{Функция активации гиперболический тангенс}
\hspace*{12.5 mm}Гиперболический тангенс (Tanh) — это сигмоидная функция, но 
симметричная относительно начала координат. Он определяется следующим образом:

\begin{equation}
    f(x) = \tanh(x) = \frac{e^x - e^{-x}}{e^x + e^{-x}}
\end{equation}

Функция принимает значения в диапазоне $(-1, 1)$, что делает её более 
предпочтительной по сравнению с сигмоидой в скрытых слоях нейронных сетей. 
Основные преимущества функции $\tanh(x)$:

1 Среднее значение её выходов ближе к нулю, что помогает ускорить сходимость 
сети.

2 Производная функции принимает большие значения в среднем диапазоне, что 
уменьшает проблему исчезающего градиента по сравнению с сигмоидой.

3 Хорошо подходит для задач, где требуется сбалансированный выход между 
положительными и отрицательными значениями.

График функции гиперболический тангенс представлен на рисунке~\ref{fig:tang}.

\insertfigure{tang}{pic/tang.png}{График функции гиперболический тангенс}{7cm}

%------------------------------------------------------------------------------
% \PART 1.2.5 Text
%------------------------------------------------------------------------------
\subsection{Обучение сети}
\hspace*{12.5 mm}Обучение FCNN осуществляется с использованием метода обратного 
распространения ошибки (Backpropagation) и оптимизационного алгоритма, 
например, стохастического градиентного спуска (SGD):

\begin{equation}
    W^{(l)} \leftarrow W^{(l)} - \eta \frac{\partial L}{\partial W^{(l)}}
\end{equation}

\begin{equation}
    b^{(l)} \leftarrow b^{(l)} - \eta \frac{\partial L}{\partial b^{(l)}}
\end{equation}

\noindentгде $\eta$ — скорость обучения, а $L$ — функция потерь, например, 
кросс-энтропия для классификации:

\begin{equation}
    L = - \sum_{i} y_i \log(\hat{y}_i)
\end{equation}

\noindentгде \( y_i \) – истинное значение, а \( \hat{y_i} \) – предсказанное 
значение.

Энтропия – измеряет степень неопределенности распределения вероятностей. А 
кросс-энтропия измеряет, насколько вероятностное распределение, предсказанное 
моделью, отличается от истинного распределения.

%------------------------------------------------------------------------------
% \PART 1.2.6 Text
%------------------------------------------------------------------------------
\subsection{Анализ полносвязной нейрооной сети}
\hspace*{12.5 mm}Структура однослойной нейронной сети прямого распространения, 
состоящей из полносвязного слоя с выходной функцией активации softmax показана 
на рисунке~\ref{fig:nn-struct}.

\insertfigure{nn-struct}{pic/NN_struct.png}{Однослойной нейронной сети прямого распространения}{10cm}

Данная архитектура позволяет добиться точности 92,4\%\cite{Doc}. Нейронная сеть 
использует 8624 параметра, которые необходимо хранить в памяти в качестве 
весовых коэффициентов и смещений.

%------------------------------------------------------------------------------
% \PART 1.3 Text
%------------------------------------------------------------------------------
\section{Сверточные нейронные сети}
\hspace*{12.5 mm}
Сверточные нейронные сети (Convolutional Neural Networks) представляют собой 
архитектуру глубокого обучения, предназначенную для обработки данных обладающих 
сетчатой топологией, таких как изображения. Их основное преимущество 
заключается в способности эффективно извлекать пространственные и иерархические 
особенности входных данных с помощью операций свертки.

Принцип работы сверточных нейронных сетей показан на рисунке~\ref{fig:cnn}\cite{CNN}.

\insertfigure{cnn}{pic/cnn.png}{Принцип работы сверточных нейронных сетей}{13cm}

%------------------------------------------------------------------------------
% \PART 1.3.1 Text
%------------------------------------------------------------------------------
\subsection{Основные компоненты сверточной нейронной сети}
\hspace*{12.5 mm}Стандартная архитектура CNN включает в себя следующие основные 
слои:

    1 \text{Сверточные слои} – выполняют операцию свертки, выделяя значимые 
признаки входных данных.

    2 \text{Функция активации} – вводит нелинейность, наиболее часто 
используется ReLU.\@

    3 \text{Слои подвыборки (Pooling)} – уменьшают размерность представления и 
повышают устойчивость к небольшим сдвигам.

    4 \text{Полносвязные слои} – выполняют классификацию извлеченных признаков.

    5 \text{Функция потерь} – используется для оценки ошибки модели, например, 
кросс-энтропия.

%------------------------------------------------------------------------------
% \PART 1.3.2 Text
%------------------------------------------------------------------------------
\subsection{Операция свертки}
\hspace*{12.5 mm}Основная операция в сверточных нейронных сетях – это свертка, 
которая формально определяется следующим образом:

\begin{equation}
    S(i, j) = \sum_{m} \sum_{n} X(i+m, j+n) \cdot K(m, n),
\end{equation}

\noindentгде:
    \( X(i, j) \) – входное изображение,
    \( K(m, n) \) – ядро свертки,
    \( S(i, j) \) – выходное изображение после свертки.

Сверточный слой применяется к входному изображению, используя несколько 
фильтров, каждый из которых извлекает определенные характеристики.

%------------------------------------------------------------------------------
% \PART 1.3.3 Text
%------------------------------------------------------------------------------
\subsection{Слои подвыборки}
\hspace*{12.5 mm}Для уменьшения размерности карт признаков применяется операция 
подвыборки. Один из наиболее распространенных методов – \text{max pooling}, 
который выбирает максимальное значение в окне \( k \times k \):

\begin{equation}
    P(i, j) = \max_{(m, n) \in k \times k} S(i+m, j+n).
\end{equation}

Этот метод помогает уменьшить объем вычислений и делает модель более устойчивой
к сдвигам изображения.

%------------------------------------------------------------------------------
% \PART 1.3.4 Text
%------------------------------------------------------------------------------
\subsection{Обучение сверточной нейронной сети}
\hspace*{12.5 mm}Обучение CNN происходит с использованием алгоритма обратного 
распространения ошибки и градиентного спуска. Для обновления параметров 
используется правило:

\begin{equation}
    w^{(t+1)} = w^{(t)} - \eta \frac{\partial L}{\partial w},
\end{equation}

\noindentгде \( \eta \) – коэффициент обучения, а \( \frac{\partial L}{\partial w} \) – 
градиент функции потерь по параметру \( w \).

%------------------------------------------------------------------------------
% \PART 1.3.5 Text
%------------------------------------------------------------------------------
\subsection{Анализ сверточной нейрооной сети}
\hspace*{12.5 mm}Структура сверточной нейронной сети, показатели точности 
которой 90\% показана на рисунке~\ref{fig:cnn-struct}\cite{CNN}.

\insertfigure{cnn-struct}{pic/CNN_STRUCT.png}{Однослойной нейронной сети прямого распространения}{10cm}

Данная конфигурации состоит из: 

  1 Входной слой: [28$\times$28]

  2 Сверточный слой: 3 маски [3$\times$3] 

  3 Слой пакетной нормализации 

  4 Слой ReLU 

  5 Полносвязанный слой 

  6 Слой Softmax 

  7 Слой классификации

Нейронная сеть использует 23520 параметра, которые необходимо хранить в памяти 
в качестве весовых коэффициентов и смещений.

%------------------------------------------------------------------------------
% \PART 1.4 Text
%------------------------------------------------------------------------------
\section{Рекуррентные нейронные сети}
\hspace*{12.5 mm}Рекуррентные нейронные сети (Recurrent Neural Networks) 
предназначены для обработки последовательных данных, таких как текст, 
аудиозаписи, временные ряды и рукописные символы. В отличие от полносвязных и 
сверточных сетей, рекуррентные обладают внутренним состоянием, позволяющим 
учитывать информацию из предыдущих шагов при обработке текущего входного 
сигнала.

Принцип работы рекуррентных нейронных сетей показан на рисунке~\ref{fig:rec}.

\insertfigure{rec}{pic/rec.png}{Принцип работы рекуррентных нейронных сетей}{11cm}

%------------------------------------------------------------------------------
% \PART 1.4.1 Text
%------------------------------------------------------------------------------
\subsection{Основная идея рекуррентных нейронных сетей}
\hspace*{12.5 mm}Основное отличие RNN от других типов нейронных сетей 
заключается в наличии обратных связей, позволяющих передавать информацию от 
предыдущих состояний. Для каждого временного шага \( t \) вычисляется новое 
состояние \( h_t \) на основе входных данных \( x_t \) и предыдущего состояния 
\( h_{t-1} \):

\begin{equation}
    h_t = f(W_x x_t + W_h h_{t-1} + b),
\end{equation}

\noindentгде:
    \( h_t \) — скрытое состояние в момент времени \( t \);
    \( x_t \) — входной вектор в момент времени \( t \);
    \( W_x \), \( W_h \) — обучаемые весовые матрицы;
    \( b \) — вектор смещения;
    \( f \) — функция активации, например \( \tanh \) или \( ReLU \).

Выход сети \( y_t \) определяется как:

\begin{equation}
    y_t = g(W_y h_t + c),
\end{equation}

\noindentгде \( W_y \) — матрица весов выходного слоя, 
\( c \) — вектор смещения, 
а \( g \) — функция активации выходного слоя.

%------------------------------------------------------------------------------
% \PART 1.4.2 Text
%------------------------------------------------------------------------------
\subsection{Проблемы стандартных рекуррентных нейронных сетей}
\hspace*{12.5 mm}Обычные RNN сталкиваются с проблемой затухающих и взрывающихся
градиентов при обучении, что делает сложным обучение долгосрочных зависимостей. 
Для решения этих проблем были разработаны улучшенные архитектуры, такие как:
долгая краткосрочная память (Long Short-Term Memory) и управляемые рекуррентные 
блоки (Gated Recurrent Unit).

Эти архитектуры используют специальные механизмы управления потоками 
информации, такие как входные, выходные и забывающие ворота, позволяя 
эффективно обрабатывать длинные последовательности данных.

%------------------------------------------------------------------------------
% \PART 1.4.3 Text
%------------------------------------------------------------------------------
\subsection{Применение RNN в распознавании рукописных цифр}
\hspace*{12.5 mm}Рекуррентные сети могут применяться для распознавания 
рукописных цифр, особенно в задачах последовательного анализа. Один из 
распространенных подходов — обработка строк рукописного текста, где каждый 
символ представлен как последовательность пикселей или векторных признаков. В 
таких задачах используются LSTM или GRU, способные учитывать 
пространственно-временные зависимости между элементами изображения. Пример 
архитектуры LSTM представлен на рисунке~\ref{fig:rnn-struct}\cite{LST2D}.

\insertfigure{rnn-struct}{pic/rnn_struct.png}{Пример архитектуры LSTM}{9cm}

Данная архитектура позволяет решать задачу распознавания рукописного текста с 
точностью 94,7\%.

%------------------------------------------------------------------------------
% \PART 1.5 Text
%------------------------------------------------------------------------------
\section{Трансформеры нейронные сети}\par
\hspace*{12.5 mm}Трансформеры представляют собой архитектуру нейронных сетей, 
которая достигла значительных успехов в обработке последовательностей данных. В
отличие от рекуррентных нейронных сетей, трансформеры используют механизмы 
самовнимания (self-attention) для обработки входной информации параллельно, что 
позволяет достигать высокой эффективности и точности.

%------------------------------------------------------------------------------
% \PART 1.5.1 Text
%------------------------------------------------------------------------------
\subsection{Архитектура трансформера}
\hspace*{12.5 mm}Архитектура трансформера состоит из энкодера и декодера, 
каждый из которых содержит несколько слоев. Основные компоненты трансформера:

  1 \text{Механизм самовнимания (self-attention)} — позволяет модели учитывать 
важность различных частей входных данных.

  2 \text{Многоголовочное внимание (multi-head attention)} — улучшает 
способность модели учитывать разные аспекты входной информации.

  3 \text{Обратная связь (residual connections)} — ускоряет обучение и 
предотвращает исчезновение градиента.

  4 \text{Механизм нормализации (layer normalization)} — стабилизирует 
обучение.

  5 \text{Feed-forward сети} — полносвязные слои, применяемые после механизма 
самовнимания.

Механизм самовнимания вычисляется следующим образом:

\begin{equation}
    \text{Attention}(Q, K, V) = \text{softmax}\left(\frac{QK^T}{\sqrt{d_k}}\right) V,
\end{equation}

\noindentгде $Q$ (query), $K$ (key) и $V$ (value) — матрицы, полученные линейными 
преобразованиями входных данных, а $d_k$ — размерность ключей.

%------------------------------------------------------------------------------
% \PART 1.5.2 Text
%------------------------------------------------------------------------------
\subsection{Преимущества трансформеров}
\hspace*{12.5 mm}В отличие от RNN, трансформеры обрабатывают входные данные 
одновременно, что ускоряет обучение. Механизм самовнимания позволяет эффективно 
учитывать контекст на больших расстояниях. Трансформеры могут применяться к 
различным задачам, включая обработку текста, изображений и речи.

%------------------------------------------------------------------------------
% \PART 1.5.3 Text
%------------------------------------------------------------------------------
\subsection{Применение трансформеров в распознавании рукописных цифр}
\hspace*{12.5 mm}Хотя трансформеры изначально были разработаны для задач 
обработки естественного языка, они также успешно применяются для распознавания 
рукописных цифр. Основной подход заключается в преобразование изображений в 
последовательности пикселей и обработка их с помощью Vision Transformer (ViT).

Трансформеры обеспечивают высокую точность  – 90,30\% и используют большое 
количество параметров – 163729 при обработке изображений, особенно при наличии 
больших объемов обучающих данных, что делает их перспективным направлением для 
распознавания рукописных цифр\cite{Agrawal}.

%------------------------------------------------------------------------------
% \PART 1.6 Text
%------------------------------------------------------------------------------
\section{Гибридные архитектуры}\par
\hspace*{12.5 mm}Гибридные архитектуры нейронных сетей представляют собой 
комбинацию различных типов нейронных сетей, объединяя их преимущества для 
улучшения производительности в задачах распознавания изображений, в том числе 
рукописных цифр. Такие модели могут включать элементы сверточных, рекуррентных, 
трансформерных и полносвязных сетей.

%------------------------------------------------------------------------------
% \PART 1.6.1 Text
%------------------------------------------------------------------------------
\subsection{Комбинация CNN и RNN}
\hspace*{12.5 mm}Одним из наиболее распространенных гибридных подходов является 
использование сверточных нейронных сетей для извлечения признаков и 
рекуррентных нейронных сетей для обработки последовательностей. 

CNN выполняет обработку изображений, извлекая пространственные признаки, 
которые затем передаются в RNN.\@

RNN, такие как LSTM или GRU, анализируют временные зависимости между 
последовательными фрагментами извлеченных признаков, что особенно полезно 
при обработке последовательностей символов или цифр.

Математически процесс можно описать следующим образом:

\begin{equation}
    F = \text{CNN}(X)
\end{equation}
\begin{equation}
    h_t = \text{RNN}(F_t, h_{t-1})
\end{equation}

\noindentгде $X$ – входное изображение, $F$ – извлеченные признаки, $h_t$ – скрытое 
состояние RNN.\@

%------------------------------------------------------------------------------
% \PART 1.6.2 Text
%------------------------------------------------------------------------------
\subsection{Сеть CNN-Transformer}
\hspace*{12.5 mm}Другой популярный гибридный подход – объединение CNN и 
трансформеров. CNN используются для локального извлечения признаков, а механизм
внимания трансформера помогает учитывать глобальный контекст изображения.

CNN генерирует карту признаков изображения.

Трансформер обрабатывает полученные признаки, используя механизм самовнимания 
(Self-Attention).

Формально, этот процесс можно записать следующим образом:

\begin{equation}
    F = \text{CNN}(X)
\end{equation}
\begin{equation}
    Z = \text{Transformer}(F)
\end{equation}

\noindentгде $Z$ – закодированное представление входного изображения, учитывающее 
пространственные и контекстные зависимости.

%------------------------------------------------------------------------------
% \PART 1.6.3 Text
%------------------------------------------------------------------------------
\subsection{Применение гибридных архитектур в распознавании рукописных цифр}
\hspace*{12.5 mm}
Гибридные модели активно применяются для задач распознавания рукописных цифр, 
поскольку они позволяют достичь высокой точности за счет сочетания мощных 
методов обработки изображений и анализа последовательностей. 

    CNN-RNN используются для распознавания рукописных цифр в 
последовательностях, таких как почтовые индексы.

    CNN-Transformer подходят для высокоточного классифицирования цифр в 
условиях зашумленных или искаженных изображений. Сеть, архитектура которой 
представлена на рисунке~\ref{fig:CNN-transformer}, позволяет добиться 
точности 99,89\%\cite{Performance}.

\insertfigure{CNN-transformer}{pic/cnn_transformer.jpeg}{Архитектура CNN-Transformer}{16cm}

Благодаря комбинированию разных методов гибридные архитектуры обеспечивают 
улучшенные результаты в задачах компьютерного зрения.               % Глава 1
%------------------------------------------------------------------------------
% \PART 2
%------------------------------------------------------------------------------
\chapter[Глава 2]{ГЛАВА 2}

%------------------------------------------------------------------------------
% \PART 2.1 Text
%------------------------------------------------------------------------------
\section{Раздел 2.1}
\hspace*{12.5 mm}Для

%------------------------------------------------------------------------------
% \PART 2.2 Text
%------------------------------------------------------------------------------
\section{Раздел 2.2}
\hspace*{12.5 mm}Устройство

Схема электрическая структурная представлена в Приложении А.               % Глава 2
%------------------------------------------------------------------------------
% \PART 3
%------------------------------------------------------------------------------
\chapter[Глава 3]
{ГЛАВА 3}

%------------------------------------------------------------------------------
% \PART 3.1 Text
%------------------------------------------------------------------------------
\section{Раздел 3.1}\par
\hspace*{12.5 mm}При 

%------------------------------------------------------------------------------
% \PART 3.2
%------------------------------------------------------------------------------
\section{Раздел 3.2}\par
\hspace*{12.5 mm}В 

\bf{3.2.1} \normalfont{Пункт без названия} 

Пример ссылки на изображение и на источник литературы:
% На рисунке~\ref{fig:esp32s} представлена распиновка ESP32S\cite{ESP-S}.

Пример вставки изображения.
% \insertfigure{esp32s}{pic/esp32s.png}{Микроконтроллер ESP32S}{9cm}
               % Глава 3
%------------------------------------------------------------------------------
% \PART 4
%------------------------------------------------------------------------------
\chapter[Разработка ...]
{РАЗРАБОТКА ..}

%------------------------------------------------------------------------------
% \PART 4.1 Text
%------------------------------------------------------------------------------
\section{Представление исходных данных}\par
\hspace*{12.5 mm}В качестве исходных данных 

%------------------------------------------------------------------------------
% \PART 4.2 Text
%------------------------------------------------------------------------------
\section{Этап 1}
\hspace*{12.5 mm}Обучение 

Пример вставки кода в текст:
\small
\fontsize{12pt}{12pt}\selectfont
\begin{verbatim}
    transform = transforms.Compose([
        transforms.ToTensor(),
        transforms.Normalize((0.5,), (0.5,))
    ])
\end{verbatim}
\normalsize
\fontsize{14pt}{14pt}\selectfont

%------------------------------------------------------------------------------
% \PART 4.3 Text
%------------------------------------------------------------------------------
\section{Программная реализация эталонной модели ...}
\hspace*{12.5 mm}Программная реализация 

               % Глава 4
%------------------------------------------------------------------------------
% \PART 5
%------------------------------------------------------------------------------
\chapter[Аппаратная реализация ...]
{АППАРАТНАЯ РЕАЛИЗАЦИЯ ..}

%------------------------------------------------------------------------------
% \PART 5.1 Text
%------------------------------------------------------------------------------
\section{Описание пакета Xilinx Vivado}
\hspace*{12.5 mm}Xilinx Vivado – это с

%------------------------------------------------------------------------------
% \PART 5.2 Text
%------------------------------------------------------------------------------
\section{Описание функциональных блоков}
\hspace*{12.5 mm}Аппаратная реализация 

%------------------------------------------------------------------------------
% \PART 5.3 Text
%------------------------------------------------------------------------------
\section{Описание интерфейса ...}
\hspace*{12.5 mm}Как было показано ранее 

%------------------------------------------------------------------------------
% \PART 5.4 Text
%------------------------------------------------------------------------------
\section{Описание процесса создания блочной диаграммы}
\hspace*{12.5 mm}Создадим блочную диаграмму

%------------------------------------------------------------------------------
% \PART 5.5 Text
%------------------------------------------------------------------------------
\section{Результаты синтеза и симуляции}
\hspace*{12.5 mm}Для проверки работоспособности                % Глава 5 
%------------------------------------------------------------------------------
% \PART 6
%------------------------------------------------------------------------------
\chapter[Технико–экономическое обоснование разработки IP–блока нейронной сети для распознавания рукописных цифр]
{ТЕХНИКО–ЭКОНОМИЧЕСКОЕ ОБОСНОВАНИЕ РАЗРАБОТКИ IP–БЛОКА НЕЙРОННОЙ СЕТИ ДЛЯ РАСПОЗНАВАНИЯ РУКОПИСНЫХ ЦИФР}

%------------------------------------------------------------------------------
% \PART 6.1 Text
%------------------------------------------------------------------------------
\section{Характеристика программного средства, разрабатываемого для собственных нужд}\par
\hspace*{12.5 mm}Разрабатываемый IP-блок нейронной сети предназначен для 
аппаратного ускорения процесса распознавания рукописных цифр в системах 
обработки изображений. Основная цель разработки — создание эффективного и 
оптимизированного аппаратного решения, интегрируемого в FPGA-платформы, для 
высокоскоростного распознавания рукописных символов с целью использования в 
автоматизированных системах ввода рукописного текста или распознавание символов 
в банковских и почтовых системах. Основные задачи, решаемые IP-блоком 
заключаются в оптимизация вычислений за счёт аппаратной реализации и 
использовании специального LST преобразования.

Разработчиком IP-блока является сотрудник IT-отдела организации. Разработка 
ведётся в рамках внутренних потребностей компании для последующего внедрения в 
производственные и исследовательские процессы.

Необходимость в разработке IP-блока обусловлена высокой потребностью в 
аппаратных решениях для распознавания рукописных цифр. Встроенные программные 
решения не обеспечивают требуемую производительность Разрабатываемое средство 
оптимизировать использование вычислительных ресурсов. Внедрение IP-блока 
позволит организации значительно повысить эффективность обработки изображений, 
что дает возможность адаптации IP-блока под специфические требования 
организации.

%------------------------------------------------------------------------------
% \PART 6.2 Text
%------------------------------------------------------------------------------
\section{Расчет инвестиций в разработку программного средства для собственных нужд}\par
\hspace*{12.5 mm}Расчет затрат на основную заработную плату представлен в 
таблице 3.

\begin{table}[ht]
    Таблица 3 – Расчет затрат на основную заработную плату команды\\\hspace*{2.6cm}разработчиков\\
    \begin{tabular}{|lccc|c|}
    \hline
    \multicolumn{1}{|c|}{\begin{tabular}[c]{@{}c@{}}Категория\\ исполнителя\end{tabular}} & \multicolumn{1}{c|}{\begin{tabular}[c]{@{}c@{}}Месячный\\ оклад, р.\end{tabular}} & \multicolumn{1}{c|}{\begin{tabular}[c]{@{}c@{}}Часовой\\ оклад, р.\end{tabular}} & \begin{tabular}[c]{@{}c@{}}Трудоемкость\\ работ, ч.\end{tabular} & Итого, р.\hspace*{4mm} \\ \hline
    \multicolumn{1}{|l|}{Бизнес-аналитик}                                                 & \multicolumn{1}{c|}{1600}                                                         & \multicolumn{1}{c|}{10}                                                          & 40                                                               & 400       \\ \hline
    \multicolumn{1}{|l|}{Системный архитектор}                                            & \multicolumn{1}{c|}{4000}                                                         & \multicolumn{1}{c|}{25}                                                          & 40                                                               & 1000      \\ \hline
    \multicolumn{1}{|l|}{Программист}                                                     & \multicolumn{1}{c|}{2400}                                                         & \multicolumn{1}{c|}{15}                                                          & 40                                                               & 600       \\ \hline
    \multicolumn{1}{|l|}{Тестировщик}                                                     & \multicolumn{1}{c|}{1600}                                                         & \multicolumn{1}{c|}{10}                                                          & 40                                                               & 400       \\ \hline
    \multicolumn{1}{|l|}{Дизайнер}                                                        & \multicolumn{1}{c|}{1600}                                                         & \multicolumn{1}{c|}{10}                                                          & 40                                                               & 400       \\ \hline
    \multicolumn{4}{|l|}{Итого}                                                                                                                                                                                                                                                                                                     & 2800      \\ \hline
    \multicolumn{4}{|l|}{Премия и иные стимулирующие выплаты (50\%)}                                                                                                                                                                                                                                                                & 1400      \\ \hline
    \multicolumn{4}{|l|}{Всего затрат на основную заработную плату разработчиков}                                                                                                                                                                                                                                                   & 4200      \\ \hline
    \end{tabular}
\end{table}

Расчет инвестиций на разработку программного средства для собственных нужд 
представлен в таблице 4.

\begin{table}[ht]
    Таблица 4 – Расчет инвестиций  на разработку программного средства для \\\hspace*{2.7cm}собственных нужд\\
    \begin{tabular}{|l|c|c|}
    \hline
    \multicolumn{1}{|c|}{Наименование статьи затрат}                                            & \begin{tabular}[c]{@{}c@{}}Формула/таблица\\ для расчета\end{tabular} & Сумма, р. \\ \hline
    \begin{tabular}[c]{@{}l@{}}1. Основная заработная плата\\ разработчиков\end{tabular}                                               & Таблица 3                                                             & 4200      \\ \hline
    \begin{tabular}[c]{@{}l@{}}2. Дополнительная заработная\\ плата разработчиков\end{tabular} & $\text{З}_\text{д} = \dfrac{4200\cdot 10\%}{100}$    &   420        \\ \hline
    \begin{tabular}[c]{@{}l@{}}3. Отчисления на социальные\\ нужды\end{tabular}                                                         & $P_{\text{соц}} = \dfrac{4200+420\cdot 34,6\%}{100}$  & 1598,52          \\ \hline
    4. Прочие расходы                                                                           & $P_{\text{пр}} = \dfrac{4200\cdot 30\%}{100}$          &   1260        \\ \hline
    \begin{tabular}[c]{@{}l@{}}5. Общая сумма инвестиций\\ на разработку\end{tabular}  &  $\text{З}_\text{р}$ = 4200+420+1598,52+1260        &    7478,52       \\ \hline
    \end{tabular}
\end{table}

%------------------------------------------------------------------------------
% \PART 6.3 Text
%------------------------------------------------------------------------------
\section{Расчет экономического эффекта от использования программного средства для собственных нужд}\par
\hspace*{12.5 mm}Экономия на заработной плате и начислениях на заработную плату 
осуществляется в результате сокращения численности работников. В частности 
после внедрения программного средства сокращению подвергается 1 дизайнер.

\noindent
\begin{equation}
    \begin{aligned}
      \text{Э}_{\text{з.п.}} = \sum_{i=1}^{n} \Delta \text{Ч}_i \cdot \text{З}_i 
      &\cdot \left(1 + \dfrac{10}{100} \right) 
      \cdot \left(1 + \dfrac{34{,}6}{100} \right) 
      = \sum_{i=1}^{1} \Delta \text{Ч}_i \cdot \text{З}_i 
      \cdot \left(1 + \dfrac{\text{Н}_{\text{д}}}{100} \right) \times \\
      \times &\left(1 + \dfrac{\text{Н}_{\text{соц}}}{100} \right) 
      = 1 \cdot 19200 \cdot 1.48 
      = 28427.52 \, \text{р}.
    \end{aligned}
\end{equation}

\noindentгде \(\text{Н}_{д}\) – норматив дополнительной заработной платы,
\(\text{Н}_{\text{соц}}\) – ставка отчисления от заработной платы, включаемых в
себестоимость.

Экономическим эффектом при использовании программного средства является прирост
чистой прибыли:

\begin{equation}
  \begin{aligned}
    \Delta\text{П}_\text{ч} &= (\text{Э}_\text{тек} - \Delta\text{З}_\text{тек}^\text{п.с})(1-\dfrac{\text{Н}_\text{п}}{100}) 
    = (28427,52 - 15500)\cdot \left(1 - \dfrac{20\%}{100}\right) =\\
    &\multicolumn{1}{c}{= 10342,02\ \text{р}.}
  \end{aligned}
\end{equation}

%------------------------------------------------------------------------------
% \PART 6.4 Text
%------------------------------------------------------------------------------
\section{Расчет показателей экономической эффективности разработки и использования программного средства в организации}\par
\hspace*{12.5 mm}Так как разработка программного средства ведется <<с нуля>>, 
то расчет показателей экономической эффективности осуществляется следующим 
образом:

\noindent
\begin{equation}
  {ROI} =  \dfrac{\Delta\text{П}_{\text{ч}} - \text{З}_{\text{р}}}{\text{З}_{\text{р}}} \cdot 100\%= \dfrac{10342,02 - 7478,52}{7478,52}\cdot 100\% = 38\%,
\end{equation}

\noindentгде $\Delta\text{П}_{\text{ч}}$ – прирост чистой прибыли, полученной
от использования разработанного программного средства, $\text{З}_{\text{р}}$ –
затраты на разработку программного средства.


На основании полученных значений показателей эффективности инвестиций (затрат) 
следует сделать вывод, что разработка IP-блока нейронной сети для аппаратного 
ускорения распознавания рукописных цифр является экономически целесообразной. 
Проведенные расчеты показывают, что общие затраты на разработку составляют 
7478,52 руб.

Экономический эффект от внедрения программного средства достигается за счет 
сокращения численности персонала (исключение 1 дизайнера), что позволяет 
сэкономить 28427,52 руб. на заработной плате и социальных начислениях. 
Прирост чистой прибыли в результате внедрения составляет 10342,02 руб.

Рассчитанный показатель ROI (рентабельность инвестиций) составляет 38\%, 
что говорит о высокой эффективности вложений в разработку данного IP-блока. 
Это означает, что вложенные средства окупаются с существенной прибылью, 
а дальнейшее использование программного средства приведет к дополнительной 
экономии и росту прибыли организации.               % Глава 6
%------------------------------------------------------------------------------
% \PART 7
%------------------------------------------------------------------------------
\chapter[Анализ результатов тестирования ...]
{АНАЛИЗ РЕЗУЛЬТАТОВ ТЕСТИРОВАНИЯ ...}

%------------------------------------------------------------------------------
% \PART 7.1 Text
%------------------------------------------------------------------------------
\section{Описание среды тестирования}\par
\hspace*{12.5 mm}Описать среду

%------------------------------------------------------------------------------
% \PART 7.2 Text
%------------------------------------------------------------------------------
\section{Тестирование разработанного ...}\par
\hspace*{12.5 mm}Кака осуществляется тестирование, скрипт на питоне.

%------------------------------------------------------------------------------
% \PART 7.3 Text
%------------------------------------------------------------------------------
\section{Экспериментальное исследование ...}\par
\hspace*{12.5 mm}Общие патерны, какие эксперименты проведены.

%------------------------------------------------------------------------------
% \PART 7.3.1 Text
%------------------------------------------------------------------------------
\subsection{Описание проводимого эксперимента}\par
\hspace*{12.5 mm}Описание конкретного эксперимента

%------------------------------------------------------------------------------
% \PART 7.3.2 Text
%------------------------------------------------------------------------------
\subsection{Подготовка данных для проведения эксперимента}\par
\hspace*{12.5 mm}Подготовка к эксперименту

%------------------------------------------------------------------------------
% \PART 7.3.3 Text
%------------------------------------------------------------------------------
\subsection{Результаты проведенного эксперимента}\
\hspace*{12.5 mm}
Результаты представлены в приложении З.

%------------------------------------------------------------------------------
% \PART 7.4 Text
%------------------------------------------------------------------------------
\section{Анализ результатов тестирования}
\hspace*{12.5 mm}сравнить результаты.
               % Глава 7
%------------------------------------------------------------------------------
% \Conclusion
%------------------------------------------------------------------------------
\section*{\vspace*{-\baselineskip}\begin{center}ЗАКЛЮЧЕНИЕ\end{center}\vspace*{-\baselineskip*2}}\par
\addcontentsline{toc}{section}{\hspace*{-1em}Заключение}\par

%------------------------------------------------------------------------------
% Text
%------------------------------------------------------------------------------
\hspace*{12.5 mm}В процессе выполнения курсового проекта по дисциплине 
«...» был           % Заключение
\clearpage % Новый лист для списка литературы
\renewcommand{\refname}{\begin{center}СПИСОК ИСПОЛЬЗОВАННЫХ ИСТОЧНИКОВ\end{center}\vspace*{-2.0em}} % Установка заголовка списка

\makeatletter
\renewcommand{\@biblabel}[1]{[#1]\hfill} % Выравнивание номера ссылки
\renewenvironment{thebibliography}[1]
     {\section*{\refname}%
      \@mkboth{\MakeUppercase\refname}{\MakeUppercase\refname}%
      \begin{list}{\@biblabel{\arabic{enumi}}}{%
          \setlength{\leftmargin}{0pt} % Убираем отступы
          \setlength{\labelwidth}{1em} % Убираем ширину метки
          \setlength{\itemindent}{1.25cm} % Убираем отступы на первой строке
          \setlength{\itemsep}{0.15em} % Убираем отступы между элементами
          \setlength{\parsep}{0pt} % Убираем дополнительный вертикальный отступ
          \usecounter{enumi}%
          \let\p@enumi\@empty
          \renewcommand{\theenumi}{\arabic{enumi}}}%
      \sloppy
      \clubpenalty4000
      \widowpenalty4000%
      \sfcode`\.\@m}
     {\end{list}}
\makeatother

\addcontentsline{toc}{section}{\hspace*{-1em}Список использованных источников}\par

\begin{thebibliography}{99}
    \setlength{\itemindent}{1.95cm} % Убираем отступы для первой строки
    \setlength{\leftmargin}{1.25cm} % Обнуляем отступы для правильного форматирования
    \setlength{\labelsep}{2mm} % Убираем стандартный отступ после номера

    \raggedright% Выравнивание по левому краю (без отступов)
    \justifying
    \bibitem{ITS2024} FPGA реализация нейронной сети прямого распространения для распознавания рукописных чисел / Е. А. Кривальцевич, М. И. Вашкевич // Информационные технологии и системы 2024 (ИТС 2024) = Information Technologies and Systems 2024 (ITS 2024) : материалы международной научной конференции, Минск, 20 ноября 2024 г. / Белорусский государственный университет информатики и радиоэлектроники; редкол. : Л. Ю. Шилин [и др.]. – Минск, 2024. – С. 85–86.
    
    \bibitem{Doc} Кривальцевич Е.А., Вашкевич М.И. Аппаратная реализация нейронной сети прямого распространения для распознавания рукописных цифр на базе FPGA. Доклады БГУИР. 20**; **(*): ***-***.
    
    \bibitem{CNN} Bouvrie J. Notes on convolutional neural networks. – 2006.
    
    \bibitem{Giardino} Giardino D. et al. FPGA implementation of hand-written number recognition based on CNN // International Journal on Advanced Science, Engineering and Information Technology. – 2019. – Т. 9. – No. 1. – P. 167-171. 
    
    \bibitem{Varsamopoulos} Varsamopoulos, S. (2019). Neural Network based Decoders for the Surface Code. [Dissertation (TU Delft), Delft University of Technology]. https://doi.org/10.4233/uuid:dc73e1ff-0496-459a-986f-de37f7f250c9
    
    \bibitem{Agrawal} Agrawal V., Jagtap J., Kantipudi M. V. V. P. Decoded-ViT: A Vision Transformer Framework for Handwritten Digit String Recognition //Revue d'Intelligence Artificielle. – 2024. – Т. 38. – №. 2. – С. 523.
    
    \bibitem{Performance} Agrawal V. et al. Performance analysis of hybrid deep learning framework using a vision transformer and convolutional neural network for handwritten digit recognition //MethodsX. – 2024. – Т. 12. – С. 102554.
    
    \bibitem{YADRO} YADRO [Электронный ресурс] – Электронные данные – Режим доступа: https://itglobal.com/ru-by/solutions/tech-partners/yadro/

    \bibitem{LST2D} Maxim Vashkevich and Egor Krivalcevich Learned 2D separable transform: building block for designing compact and efficient neural network for image recognition.
    
    \bibitem{MNIST} MNIST Handwritten Numbers database [Electronic resource] – Electronic data – Access mode: https://yann.lecun.com/exdb/mnist/Mittal

    \bibitem{vivado_axi} Xilinx, {AXI Reference Guide} [Электронный ресурс] – Электронные данные – Режим доступа: https://docs.amd.com/v/u/en-US/ug1037-vivado-axi-reference-guide, 2017.

    \bibitem{chu_fpga} Pong P. Chu, {FPGA Prototyping by Verilog Examples} [Электронный ресурс] – Электронные данные – Режим доступа: https://faculty.kfupm.edu.sa/COE/aimane/coe405/FPGA\%20examples.pdf, 2018.

    \bibitem{Tokheim2003} Tokheim, R.L. Digital Electronics: Principles and Applications. – McGraw-Hill, 2003. – 480 p.

    \bibitem{Katz1994} Katz, R.H. Contemporary Logic Design. – Benjamin-Cummings, 1994. – 700 p.

    \bibitem{XilinxUG} Xilinx Inc., \textit{Vivado Design Suite User Guide}, UG892, 2021.  

    \bibitem{TANH}  L. D. Medus, T. Iakymchuk, J. V. Frances-Villora, M. Bataller-
Mompean, and A. Rosado-Munoz, “A novel systolic parallel hardware
architecture for the FPGA acceleration of feedforward neural networks,”
IEEE Access, vol. 7, pp. 76 084–76 103, 2019.


\end{thebibliography}
              % Библиография 

% -----------------------------------------------------------------------------
% ПРИЛОЖЕНИЯ
% -----------------------------------------------------------------------------
\clearpage
\phantomsection% ensures hyperref works properly
\addcontentsline{toc}{section}{\hspace*{-1em}Приложение А (Обязательное) Отчет о проверке на заимствование}\par

\normalsize
\begin{center}
  \textbf{ПРИЛОЖЕНИЕ А} \\
  \textbf{(Обязательное)} \\
  \textbf{Отчет о проверке на заимствование}
\end{center}

\clearpage

% Установка размера страницы A3 в альбомной ориентации
% \pdfpagewidth=420mm
% \pdfpageheight=297mm

\thispagestyle{empty} % Убрать номер страницы

% Вставка листа A3
% \includepdf[pages=-, pagecommand={}, fitpaper, offset=0 0]{STRUCT.pdf}

% Восстановление размера страницы A4
\pdfpagewidth=210mm
\pdfpageheight=297mm

\clearpage
               % Приложение А 
% Ваш основной текст
\clearpage
\phantomsection% ensures hyperref works properly
\addcontentsline{toc}{section}{\hspace*{-1em}Приложение Б (Обязательное) Схема электрическая структурная LST-1}\par

\normalsize
\begin{center}
  \textbf{ПРИЛОЖЕНИЕ Б} \\
  \textbf{(Обязательное)} \\
  \textbf{Схема электрическая структурная LST-1}
\end{center}

\clearpage

% Установка размера страницы A3 в альбомной ориентации
% \pdfpagewidth=420mm
% \pdfpageheight=297mm

\thispagestyle{empty} % Убрать номер страницы

% Вставка листа A3
\includepdf[pages=-, pagecommand={}, fitpaper, offset=0 0]{STRUCT_LST.pdf}

% Восстановление размера страницы A4
\pdfpagewidth=210mm
\pdfpageheight=297mm

\clearpage
               % Приложение Б
% Ваш основной текст
\clearpage
\phantomsection% ensures hyperref works properly
\addcontentsline{toc}{section}{\hspace*{-1em}Приложение В (Обязательное) Пример вставки 2 pdf}\par

\normalsize
\begin{center}
  \textbf{ПРИЛОЖЕНИЕ В} \\
  \textbf{(Обязательное)} \\
  \textbf{Пример вставки 2 pdf}
\end{center}

\clearpage

% Установка размера страницы A3 в альбомной ориентации
% \pdfpagewidth=420mm
% \pdfpageheight=297mm

\thispagestyle{empty} % Убрать номер страницы

% Вставка листа A3
% \includepdf[pages=-, pagecommand={}, fitpaper, offset=0 0]{prilC.pdf}
% \thispagestyle{empty} % Убрать номер страницы
% \includepdf[pages=-, pagecommand={}, fitpaper, offset=0 0]{prilC1.pdf}

% Восстановление размера страницы A4
\pdfpagewidth=210mm
\pdfpageheight=297mm
\clearpage
               % Приложение В 
\clearpage
\phantomsection% ensures hyperref works properly
\addcontentsline{toc}{section}{\hspace*{-1em}Приложение Г (Обязательное) C описание устройства}\par

\normalsize
\begin{center}
  \textbf{ПРИЛОЖЕНИЕ Г} \\
  \textbf{(Обязательное)} \\
  \textbf{C описание устройства}
\end{center}

Модуль инициализации камеры:

\lstinputlisting[language=C]{C/camera_setup.h}

Модуль выполнения основных функций:

\lstinputlisting[language=C]{C/web_server.h}

Модуль сборки всего проекта:

\lstinputlisting[language=C]{C/camera.ino}

% Восстановление размера страницы A4
\pdfpagewidth=210mm
\pdfpageheight=297mm

\clearpage
               % Приложение Г 
\clearpage
\phantomsection% ensures hyperref works properly
\addcontentsline{toc}{section}{\hspace*{-1em}Приложение Д (Обязательное) Пример использования Python}\par

\normalsize
\begin{center}
  \textbf{ПРИЛОЖЕНИЕ Д} \\
  \textbf{(Обязательное)} \\
  \textbf{Пример использования Python}
\end{center}

Пример использования Python:

% \lstinputlisting[language=python]{C/learn.py}


% Восстановление размера страницы A4
\pdfpagewidth=210mm
\pdfpageheight=297mm               % Приложение Д 
\include{prilF}               % Приложение Е 
\include{prilG}               % Приложение Ж 
\clearpage
\phantomsection% ensures hyperref works properly
\addcontentsline{toc}{section}{\hspace*{-1em}Приложение З (Обязательное) Пример использования verilog}\par

\normalsize
\begin{center}
  \textbf{ПРИЛОЖЕНИЕ З} \\
  \textbf{(Обязательное)} \\
  \textbf{Пример использования verilog}
\end{center}

% \clearpage

Пример использования verilog:

% \lstinputlisting[language=Verilog]{C/softmax.sv}

% Восстановление размера страницы A4
\pdfpagewidth=210mm
\pdfpageheight=297mm

% \clearpage
               % Приложение З 
\clearpage
\phantomsection% ensures hyperref works properly
\addcontentsline{toc}{section}{\hspace*{-1em}Приложение И (Обязательное) Результаты работы нейронной сети}\par

\normalsize
\begin{center}
  \textbf{ПРИЛОЖЕНИЕ И} \\
  \textbf{(Обязательное)} \\
  \textbf{Результаты работы нейронной сети}
\end{center}

\clearpage

% Установка размера страницы A3 в альбомной ориентации
% \pdfpagewidth=420mm
% \pdfpageheight=297mm

\thispagestyle{empty} % Убрать номер страницы

% Вставка листа A3
% \includepdf[pages=-, pagecommand={}, fitpaper, offset=0 0]{ALG.pdf}

% Восстановление размера страницы A4
% \pdfpagewidth=210mm
% \pdfpageheight=297mm
% \includepdf[pages=-, pagecommand={}, fitpaper, offset=0 0]{perechen.pdf}
% \clearpage
               % Приложение И      

% \clearpage

\thispagestyle{empty} % Убрать номер страницы

% Вставка листа A4
\includepdf[pages=-, pagecommand={}, fitpaper, offset=0 0]{vedoma.pdf}

% Восстановление размера страницы A4
\pdfpagewidth=210mm
\pdfpageheight=297mm

\clearpage


\end{document}
