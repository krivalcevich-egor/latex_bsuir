%------------------------------------------------------------------------------
% \PART 6
%------------------------------------------------------------------------------
\chapter[Технико–экономическое обоснование разработки IP–блока нейронной сети для распознавания рукописных цифр]
{ТЕХНИКО–ЭКОНОМИЧЕСКОЕ ОБОСНОВАНИЕ РАЗРАБОТКИ IP–БЛОКА НЕЙРОННОЙ СЕТИ ДЛЯ РАСПОЗНАВАНИЯ РУКОПИСНЫХ ЦИФР}

%------------------------------------------------------------------------------
% \PART 6.1 Text
%------------------------------------------------------------------------------
\section{Характеристика программного средства, разрабатываемого для собственных нужд}\par
\hspace*{12.5 mm}Разрабатываемый IP-блок нейронной сети предназначен для 
аппаратного ускорения процесса распознавания рукописных цифр в системах 
обработки изображений. Основная цель разработки — создание эффективного и 
оптимизированного аппаратного решения, интегрируемого в FPGA-платформы, для 
высокоскоростного распознавания рукописных символов с целью использования в 
автоматизированных системах ввода рукописного текста или распознавание символов 
в банковских и почтовых системах. Основные задачи, решаемые IP-блоком 
заключаются в оптимизация вычислений за счёт аппаратной реализации и 
использовании специального LST преобразования.

Разработчиком IP-блока является сотрудник IT-отдела организации. Разработка 
ведётся в рамках внутренних потребностей компании для последующего внедрения в 
производственные и исследовательские процессы.

Необходимость в разработке IP-блока обусловлена высокой потребностью в 
аппаратных решениях для распознавания рукописных цифр. Встроенные программные 
решения не обеспечивают требуемую производительность Разрабатываемое средство 
оптимизировать использование вычислительных ресурсов. Внедрение IP-блока 
позволит организации значительно повысить эффективность обработки изображений, 
что дает возможность адаптации IP-блока под специфические требования 
организации.

%------------------------------------------------------------------------------
% \PART 6.2 Text
%------------------------------------------------------------------------------
\section{Расчет инвестиций в разработку программного средства для собственных нужд}\par
\hspace*{12.5 mm}Расчет затрат на основную заработную плату представлен в 
таблице 3.

\begin{table}[ht]
    Таблица 3 – Расчет затрат на основную заработную плату команды\\\hspace*{2.6cm}разработчиков\\
    \begin{tabular}{|lccc|c|}
    \hline
    \multicolumn{1}{|c|}{\begin{tabular}[c]{@{}c@{}}Категория\\ исполнителя\end{tabular}} & \multicolumn{1}{c|}{\begin{tabular}[c]{@{}c@{}}Месячный\\ оклад, р.\end{tabular}} & \multicolumn{1}{c|}{\begin{tabular}[c]{@{}c@{}}Часовой\\ оклад, р.\end{tabular}} & \begin{tabular}[c]{@{}c@{}}Трудоемкость\\ работ, ч.\end{tabular} & Итого, р.\hspace*{4mm} \\ \hline
    \multicolumn{1}{|l|}{Бизнес-аналитик}                                                 & \multicolumn{1}{c|}{1600}                                                         & \multicolumn{1}{c|}{10}                                                          & 40                                                               & 400       \\ \hline
    \multicolumn{1}{|l|}{Системный архитектор}                                            & \multicolumn{1}{c|}{4000}                                                         & \multicolumn{1}{c|}{25}                                                          & 40                                                               & 1000      \\ \hline
    \multicolumn{1}{|l|}{Программист}                                                     & \multicolumn{1}{c|}{2400}                                                         & \multicolumn{1}{c|}{15}                                                          & 40                                                               & 600       \\ \hline
    \multicolumn{1}{|l|}{Тестировщик}                                                     & \multicolumn{1}{c|}{1600}                                                         & \multicolumn{1}{c|}{10}                                                          & 40                                                               & 400       \\ \hline
    \multicolumn{1}{|l|}{Дизайнер}                                                        & \multicolumn{1}{c|}{1600}                                                         & \multicolumn{1}{c|}{10}                                                          & 40                                                               & 400       \\ \hline
    \multicolumn{4}{|l|}{Итого}                                                                                                                                                                                                                                                                                                     & 2800      \\ \hline
    \multicolumn{4}{|l|}{Премия и иные стимулирующие выплаты (50\%)}                                                                                                                                                                                                                                                                & 1400      \\ \hline
    \multicolumn{4}{|l|}{Всего затрат на основную заработную плату разработчиков}                                                                                                                                                                                                                                                   & 4200      \\ \hline
    \end{tabular}
\end{table}

Расчет инвестиций на разработку программного средства для собственных нужд 
представлен в таблице 4.

\begin{table}[ht]
    Таблица 4 – Расчет инвестиций  на разработку программного средства для \\\hspace*{2.7cm}собственных нужд\\
    \begin{tabular}{|l|c|c|}
    \hline
    \multicolumn{1}{|c|}{Наименование статьи затрат}                                            & \begin{tabular}[c]{@{}c@{}}Формула/таблица\\ для расчета\end{tabular} & Сумма, р. \\ \hline
    \begin{tabular}[c]{@{}l@{}}1. Основная заработная плата\\ разработчиков\end{tabular}                                               & Таблица 3                                                             & 4200      \\ \hline
    \begin{tabular}[c]{@{}l@{}}2. Дополнительная заработная\\ плата разработчиков\end{tabular} & $\text{З}_\text{д} = \dfrac{4200\cdot 10\%}{100}$    &   420        \\ \hline
    \begin{tabular}[c]{@{}l@{}}3. Отчисления на социальные\\ нужды\end{tabular}                                                         & $P_{\text{соц}} = \dfrac{4200+420\cdot 34,6\%}{100}$  & 1598,52          \\ \hline
    4. Прочие расходы                                                                           & $P_{\text{пр}} = \dfrac{4200\cdot 30\%}{100}$          &   1260        \\ \hline
    \begin{tabular}[c]{@{}l@{}}5. Общая сумма инвестиций\\ на разработку\end{tabular}  &  $\text{З}_\text{р}$ = 4200+420+1598,52+1260        &    7478,52       \\ \hline
    \end{tabular}
\end{table}

%------------------------------------------------------------------------------
% \PART 6.3 Text
%------------------------------------------------------------------------------
\section{Расчет экономического эффекта от использования программного средства для собственных нужд}\par
\hspace*{12.5 mm}Экономия на заработной плате и начислениях на заработную плату 
осуществляется в результате сокращения численности работников. В частности 
после внедрения программного средства сокращению подвергается 1 дизайнер.

\noindent
\begin{equation}
    \begin{aligned}
      \text{Э}_{\text{з.п.}} = \sum_{i=1}^{n} \Delta \text{Ч}_i \cdot \text{З}_i 
      &\cdot \left(1 + \dfrac{10}{100} \right) 
      \cdot \left(1 + \dfrac{34{,}6}{100} \right) 
      = \sum_{i=1}^{1} \Delta \text{Ч}_i \cdot \text{З}_i 
      \cdot \left(1 + \dfrac{\text{Н}_{\text{д}}}{100} \right) \times \\
      \times &\left(1 + \dfrac{\text{Н}_{\text{соц}}}{100} \right) 
      = 1 \cdot 19200 \cdot 1.48 
      = 28427.52 \, \text{р}.
    \end{aligned}
\end{equation}

\noindentгде \(\text{Н}_{д}\) – норматив дополнительной заработной платы,
\(\text{Н}_{\text{соц}}\) – ставка отчисления от заработной платы, включаемых в
себестоимость.

Экономическим эффектом при использовании программного средства является прирост
чистой прибыли:

\begin{equation}
  \begin{aligned}
    \Delta\text{П}_\text{ч} &= (\text{Э}_\text{тек} - \Delta\text{З}_\text{тек}^\text{п.с})(1-\dfrac{\text{Н}_\text{п}}{100}) 
    = (28427,52 - 15500)\cdot \left(1 - \dfrac{20\%}{100}\right) =\\
    &\multicolumn{1}{c}{= 10342,02\ \text{р}.}
  \end{aligned}
\end{equation}

%------------------------------------------------------------------------------
% \PART 6.4 Text
%------------------------------------------------------------------------------
\section{Расчет показателей экономической эффективности разработки и использования программного средства в организации}\par
\hspace*{12.5 mm}Так как разработка программного средства ведется <<с нуля>>, 
то расчет показателей экономической эффективности осуществляется следующим 
образом:

\noindent
\begin{equation}
  {ROI} =  \dfrac{\Delta\text{П}_{\text{ч}} - \text{З}_{\text{р}}}{\text{З}_{\text{р}}} \cdot 100\%= \dfrac{10342,02 - 7478,52}{7478,52}\cdot 100\% = 38\%,
\end{equation}

\noindentгде $\Delta\text{П}_{\text{ч}}$ – прирост чистой прибыли, полученной
от использования разработанного программного средства, $\text{З}_{\text{р}}$ –
затраты на разработку программного средства.


На основании полученных значений показателей эффективности инвестиций (затрат) 
следует сделать вывод, что разработка IP-блока нейронной сети для аппаратного 
ускорения распознавания рукописных цифр является экономически целесообразной. 
Проведенные расчеты показывают, что общие затраты на разработку составляют 
7478,52 руб.

Экономический эффект от внедрения программного средства достигается за счет 
сокращения численности персонала (исключение 1 дизайнера), что позволяет 
сэкономить 28427,52 руб. на заработной плате и социальных начислениях. 
Прирост чистой прибыли в результате внедрения составляет 10342,02 руб.

Рассчитанный показатель ROI (рентабельность инвестиций) составляет 38\%, 
что говорит о высокой эффективности вложений в разработку данного IP-блока. 
Это означает, что вложенные средства окупаются с существенной прибылью, 
а дальнейшее использование программного средства приведет к дополнительной 
экономии и росту прибыли организации.