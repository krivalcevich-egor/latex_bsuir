\clearpage % Новый лист для списка литературы
\renewcommand{\refname}{\begin{center}СПИСОК ИСПОЛЬЗОВАННЫХ ИСТОЧНИКОВ\end{center}\vspace*{-2.0em}} % Установка заголовка списка

\makeatletter
\renewcommand{\@biblabel}[1]{[#1]\hfill} % Выравнивание номера ссылки
\renewenvironment{thebibliography}[1]
     {\section*{\refname}%
      \@mkboth{\MakeUppercase\refname}{\MakeUppercase\refname}%
      \begin{list}{\@biblabel{\arabic{enumi}}}{%
          \setlength{\leftmargin}{0pt} % Убираем отступы
          \setlength{\labelwidth}{1em} % Убираем ширину метки
          \setlength{\itemindent}{1.25cm} % Убираем отступы на первой строке
          \setlength{\itemsep}{0.15em} % Убираем отступы между элементами
          \setlength{\parsep}{0pt} % Убираем дополнительный вертикальный отступ
          \usecounter{enumi}%
          \let\p@enumi\@empty
          \renewcommand{\theenumi}{\arabic{enumi}}}%
      \sloppy
      \clubpenalty4000
      \widowpenalty4000%
      \sfcode`\.\@m}
     {\end{list}}
\makeatother

\addcontentsline{toc}{section}{\hspace*{-1em}Список использованных источников}\par

\begin{thebibliography}{99}
    \setlength{\itemindent}{1.95cm} % Убираем отступы для первой строки
    \setlength{\leftmargin}{1.25cm} % Обнуляем отступы для правильного форматирования
    \setlength{\labelsep}{2mm} % Убираем стандартный отступ после номера

    \raggedright% Выравнивание по левому краю (без отступов)
    \justifying
    \bibitem{ITS2024} FPGA реализация нейронной сети прямого распространения для распознавания рукописных чисел / Е. А. Кривальцевич, М. И. Вашкевич // Информационные технологии и системы 2024 (ИТС 2024) = Information Technologies and Systems 2024 (ITS 2024) : материалы международной научной конференции, Минск, 20 ноября 2024 г. / Белорусский государственный университет информатики и радиоэлектроники; редкол. : Л. Ю. Шилин [и др.]. – Минск, 2024. – С. 85–86.

\end{thebibliography}
